\documentclass[fleqn]{article}
\oddsidemargin 0.0in
\textwidth 6.0in
\thispagestyle{empty}
\usepackage{import}
\usepackage{amsmath}
\usepackage{graphicx}
\usepackage[english]{babel}
\usepackage[utf8x]{inputenc}
\usepackage{float}
\usepackage[colorinlistoftodos]{todonotes}
\usepackage{mathtools}
\usepackage[thinc]{esdiff}

\definecolor{hwColor}{HTML}{AD53BA}

\begin{document}

  \begin{titlepage}

    \newcommand{\HRule}{\rule{\linewidth}{0.5mm}} % Defines a new command for the horizontal lines, change thickness here

    \center % Center everything on the page
    

    \textsc{\LARGE Arizona State University}\\[1.5cm]

    \textsc{\LARGE Mathematical Methods For Physics I }\\[1.5cm]


    \begin{figure}
      \includegraphics[width=\linewidth]{asu.png}
    \end{figure}


    \HRule \\[0.4cm]
    { \huge \bfseries Homework 10}\\[0.4cm] 
    \HRule \\[1.5cm]
    
    \textbf{Behnam Amiri}

    \bigbreak

    \textbf{Prof: Cecilia Lunardini}

    \bigbreak


    \textbf{{\large \today}\\[2cm]}

    \vfill % Fill the rest of the page with whitespace

  \end{titlepage}

  \begin{enumerate}
    \item  Examine the equations below, where $y(x)$ is a real function of a real variable $x$, and $y^\prime=\diff{y}{x}$,  $y^{\prime\prime}=\diff[2]{y}{x}$, etc. For each case, establish if the equation is separable, and if it is, solve it (pay attention: because no initial conditions are given, each solution will contain an undetermined constant). 
    \begin{equation*}
      a. ~~ \dfrac{dy}{dx}+2xy^2=0
    \end{equation*}
    \begin{equation*}
      b. ~~ \dfrac{y^{\prime}}{y+1}=3
    \end{equation*}
    \begin{equation*}
      c. ~~ y^{\prime}+3y=e^{2x}
    \end{equation*}
    \begin{equation*}
      d. ~~ y^{\prime}=\dfrac{y}{x}
    \end{equation*}
    \begin{equation*}
      e. ~~ e^{x+y}y^{\prime}=x
    \end{equation*}
    \begin{equation*}
      f. ~~ (x+y)y^{\prime}=x-y
    \end{equation*}
    \begin{equation*}
      g. ~~ \dfrac{y^{\prime}-y}{x}=\dfrac{1}{1+x^2}
    \end{equation*}

    \item Prove that an equation of the form:
    $$
    \diff{y}{x} + y P(x) = 0
    $$
    is always separable and find the expression of its general solution.  

    \item Derive eq. (14.12) of the textbook (start from eq. (14/10) ). 
    

    \item Show that the following equations have the form $dU(x,y)=0$ (i.e., the left side of the equation is an exact differential). For each equation, find the general solution and the specific solution that satisfies the initial condition given.
    \begin{equation*}
      a. ~~ (2y+3x)dy+(3x^2+3y)dx=0, ~~~ y(0)=1
    \end{equation*}
    \begin{equation*}
      b. ~~ (3x^2y^2+e^y)dy+2(xy^3+1)dx=0, ~~~ y(0)=2
    \end{equation*}
    \begin{equation*}
      c. ~~ (xe^y-1)dy+e^ydx=0 ~~~ y(0)=2
    \end{equation*}
    
    \item Solve each of the following differential equations using the integrating factor method (regardless of whether they are solvable by other means) to render them exact:
    \begin{equation*}
      a. ~~ 2xdy+y(1+x)dx=0
    \end{equation*}
    \begin{equation*}
      b. ~~ cos(x-y)y^{\prime}-cos(x-y)-sin(x-y)=0
    \end{equation*}
    \begin{equation*}
      c. ~~  (y+e^x)y^{\prime}+2ye^x+\dfrac{y^2}{2}=0
    \end{equation*}
    \begin{equation*}
      d. ~~ (y+x)y^{\prime}+\dfrac{y-1}{x}+2+e^{-y}=0
    \end{equation*}
  \end{enumerate}

\end{document}