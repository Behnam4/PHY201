\documentclass[fleqn]{article}
\oddsidemargin 0.0in
\textwidth 6.0in
\thispagestyle{empty}
\usepackage{import}
\usepackage{amsmath}
\usepackage{bigints}
\usepackage{graphicx}
\usepackage[english]{babel}
\usepackage[utf8x]{inputenc}
\usepackage{float}
\usepackage[colorinlistoftodos]{todonotes}
\usepackage{mathtools}
\usepackage[thinc]{esdiff}

\definecolor{hwColor}{HTML}{AD53BA}

\begin{document}

  \begin{titlepage}

    \newcommand{\HRule}{\rule{\linewidth}{0.5mm}} % Defines a new command for the horizontal lines, change thickness here

    \center % Center everything on the page
    

    \textsc{\LARGE Arizona State University}\\[1.5cm]

    \textsc{\LARGE Mathematical Methods For Physics I }\\[1.5cm]


    \begin{figure}
      \includegraphics[width=\linewidth]{asu.png}
    \end{figure}


    \HRule \\[0.4cm]
    { \huge \bfseries Homework 10}\\[0.4cm] 
    \HRule \\[1.5cm]
    
    \textbf{Behnam Amiri}

    \bigbreak

    \textbf{Prof: Cecilia Lunardini}

    \bigbreak


    \textbf{{\large \today}\\[2cm]}

    \vfill % Fill the rest of the page with whitespace

  \end{titlepage}

  \begin{enumerate}
    \item  Examine the equations below, where $y(x)$ is a real function of a real variable $x$, and $y^\prime=\diff{y}{x}$,  $y^{\prime\prime}=\diff[2]{y}{x}$, etc. For each case, establish if the equation is separable, and if it is, solve it (pay attention: because no initial conditions are given, each solution will contain an undetermined constant). 
    \begin{equation*}
      a. ~~ \dfrac{dy}{dx}+2xy^2=0
    \end{equation*}
    \begin{equation*}
      b. ~~ \dfrac{y^{\prime}}{y+1}=3
    \end{equation*}
    \begin{equation*}
      c. ~~ y^{\prime}+3y=e^{2x}
    \end{equation*}
    \begin{equation*}
      d. ~~ y^{\prime}=\dfrac{y}{x}
    \end{equation*}
    \begin{equation*}
      e. ~~ e^{x+y}y^{\prime}=x
    \end{equation*}
    \begin{equation*}
      f. ~~ (x+y)y^{\prime}=x-y
    \end{equation*}
    \begin{equation*}
      g. ~~ \dfrac{y^{\prime}-y}{x}=\dfrac{1}{1+x^2}
    \end{equation*}

      \textcolor{hwColor}{ 
        $a. ~~ \diff{y}{x}+2xy^2=0 \rightarrow \diff{y}{x}=-2xy^2 \rightarrow \dfrac{dy}{y^2}=-2xdx$ \\
        $\bigints \dfrac{dy}{y^2}=\bigints -2xdx$ \\
        $-\dfrac{1}{y}=-x^2+C$ \\
        $\Longrightarrow y(x)=\dfrac{1}{x^2-C}$
      } 

      \textcolor{hwColor}{  
        \rule{15cm}{0.4pt}  
      }

      \textcolor{hwColor}{ 
        $b. ~~ \dfrac{y^{\prime}}{y+1}=3 \rightarrow \diff{y}{x}=3(y+1) \rightarrow \dfrac{dy}{y+1}=3dx$ \\
        $\bigints \dfrac{dy}{y+1}=\bigints 3dx$ \\
        $ln|y+1|=3x+C \rightarrow |y+1|=e^{3x+C} \rightarrow y+1=\pm e^{3x+C}$ \\
        \\
        $\Longrightarrow y(x)=\left[\pm e^{3x+C}\right]-1$
      }
      
      \textcolor{hwColor}{  
        \rule{15cm}{0.4pt}  
      } 

      \textcolor{hwColor}{ 
        $c. ~~ y^{\prime}+3y=e^{2x} \rightarrow \diff{y}{x}+3y=e^{2x}$ \\
        Multiple both sides by $e^{\bigints 3dx}$ \\
        \\
        $e^{3x}\diff{y}{x}+e^{3x}3y=e^{3x} e^{2x} \rightarrow D_x\left[e^{3x}y\right]=e^{5x}$ \\
        $\bigints D_x\left[e^{3x}y\right]dx=\bigints e^{5x}dx$ \\
        $e^{3x}y=\dfrac{1}{5}e^{5x}+C \rightarrow y=\dfrac{\dfrac{1}{5}e^{5x}+C}{e^{3x}}$ \\
        $\Longrightarrow y(x)=\dfrac{1}{5}e^{2x}+\dfrac{C}{e^{3x}}$
      }

      \textcolor{hwColor}{  
        \rule{15cm}{0.4pt}  
      }

      \textcolor{hwColor}{ 
        $d. ~~ y^{\prime}=\dfrac{y}{x} \rightarrow \diff{y}{x}=\dfrac{y}{x} \rightarrow xdy=ydx$ \\
        $\dfrac{dy}{y}=\dfrac{dx}{x} \rightarrow \bigints \dfrac{dy}{y}=\bigints \dfrac{dx}{x}$ \\
        $ln|y|=ln|x|+C \rightarrow |y|=e^{ln|x|+C}=e^{ln|x|}.e^{C}=x.e^C$ \\
        $\Longrightarrow y(x)=\pm x.e^{C}$
      }

      \textcolor{hwColor}{  
        \rule{15cm}{0.4pt}  
      }

      \textcolor{hwColor}{ 
        $e. ~~ e^{x+y}y^{\prime}=x \rightarrow e^xe^y \dfrac{dy}{dx}=x \rightarrow e^ydy=xe^{-x}dx$ \\
        $\bigints e^ydy=\bigints xe^{-x}dx=-e^{-x}x-\bigints -e^{-x}dx=-e^{-x}x+\bigints e^{-x}dx=-e^{-x}x-e^{-x}+C$ \\
        $e^y=-e^{-x}x-e^{-x}+C \rightarrow lne^y=ln(-e^{-x}x-e^{-x}+C)$ \\
        $\Longrightarrow y(x)=ln(-e^{-x}x-e^{-x}+C)$
      }

      \textcolor{hwColor}{  
        \rule{15cm}{0.4pt}  
      }

      \textcolor{hwColor}{ 
        $f. ~~ (x+y)y^{\prime}=x-y \rightarrow (-x+y)+(x+y)\diff{y}{x}=0 \rightarrow (-x+y)dx+(x+y)dy=0$ \\
        $
          \begin{cases}
            \dfrac{\partial F}{\partial x}=F_x=M=-x+y \\
            ~~~~~~~~~~~~~~~~~~~~~~~~~~~~~~~~~~~~~~~~ \rightarrow \dfrac{\partial M}{\partial y}=\dfrac{\partial N}{\partial x} \Longrightarrow 1=1 \\
            \dfrac{\partial F}{\partial y}=F_y=N=x+y
          \end{cases}
        $ \\
        \\
        This means we have an exact differential equation for some higher parent function. \\
        $M=\dfrac{\partial F}{\partial x}=-x+y \rightarrow \bigints \dfrac{\partial F}{\partial x} dx=\bigints (-x+y)dx \rightarrow F(x+y)=-\dfrac{x^2}{2}+xy+g(y)$ \\
        $\dfrac{\partial F}{\partial y}=0+x+g^{\prime}(y)$ \\
        $x+y=x+g^{\prime}(y) \rightarrow g^{\prime}(y)=y \Longrightarrow g(y)=\bigints ydy=\dfrac{y^2}{2}$ \\
        $F(x,y)=-\dfrac{x^2}{2}+xy+\dfrac{y^2}{2}=C$
      }

      \textcolor{hwColor}{  
        \rule{15cm}{0.4pt}  
      }

      \textcolor{hwColor}{ 
        $g. ~~ \dfrac{y^{\prime}-y}{x}=\dfrac{1}{1+x^2} \rightarrow \diff{y}{x}-y=\dfrac{x}{1+x^2}$ \\
        Multiple both sides by $e^{\bigints -dx}$ \\
        \\
        $e^{-x} \diff{y}{x}-e^{-x}y=e^{-x}\dfrac{x}{1+x^2}$ \\
        $D_x\left[e^{-x}y\right]=e^{-x}\dfrac{x}{1+x^2}$ \\
        $\bigints D_x\left[e^{-x}y\right]dx=\bigints e^{-x}\dfrac{x}{1+x^2}dx$ \\
        $e^{-x}y=\bigints e^{-x}\dfrac{x}{1+x^2}dx$ \\
        $y(x)=e^x \bigints e^{-x}\dfrac{x}{1+x^2}dx$
      }

    \item Prove that an equation of the form:
    $$
    \diff{y}{x} + y P(x) = 0
    $$
    is always separable and find the expression of its general solution. \\
      \textcolor{hwColor}{ 
        $
          \diff{y}{x}+yP(x)=0 \rightarrow \diff{y}{x}=-yP(x) \rightarrow \dfrac{dy}{ydx}=-P(x)
        $
        \\
        $\dfrac{dy}{y}=-P(x)dx \rightarrow \bigints \dfrac{dy}{y}=\bigints -P(x)dx$
        \\
        $ln|y|=-\bigints P(x)dx$
        \\
        $y(x)=\pm e^{-\bigints P(x)dx}$ \\
        Assume $\bigints P(x)dx=k+C$ where $C$ is a constant.  \\
        \\
        Then $y(x)=\pm e^{-(k+C)}=\pm e^{-k}.e^C$
        \\
        $e^C$ is going to be a constant. Hence, suppose $e^C=W$ \\
        \\
        $\Longrightarrow y(x)=\pm W.e^{-k}$    
      }

    \item Derive eq. (14.12) of the textbook (start from eq. (14/10) ).
      \textcolor{hwColor}{ 
        $
          F(x,y)=C \\
          \dfrac{d}{dx}\left[F(x,y)=C\right] \rightarrow \diff{F}{x}=\dfrac{\partial F}{\partial x}+\dfrac{\partial F}{\partial y}.\dfrac{\partial y}{\partial x}=0 \\
          \\
          dF=\dfrac{\partial F}{\partial x}dx+\dfrac{\partial F}{\partial y}dy=0 \rightarrow M=\dfrac{\partial F}{\partial x} ~~~~ N=\dfrac{\partial F}{\partial y} \\
          dF=M dx+N dy=0
        $ \\
        When $\dfrac{\partial M}{\partial y}=\dfrac{\partial N}{\partial x}$, we have an exact differential equation. \\
        \\
        If $M_y \ne N_x$ then we have an unexact differential equation. let's see if we can multiple both sides of $dF$ by a function to make it an exact differential equation. \\
        $
          u(x,y)\left[M dx+N dy=0\right] \rightarrow u(x,y)M dx+u(x,y)N dy=0
        $ \\
        We want $\dfrac{\partial}{\partial y}\left[u(x,y).M\right]=\dfrac{\partial }{\partial x}\left[u(x,y).N\right]$ to guarantee that is an exact differential equation. \\
        \\
        $
          \dfrac{\partial u}{\partial y}.M+u.\dfrac{\partial M}{\partial y}=\dfrac{\partial u}{\partial x}.N+u.\dfrac{\partial N}{\partial x}\\
          \\
          u\left[\dfrac{\partial M}{\partial y}-\dfrac{\partial N}{\partial x}\right]=\dfrac{\partial u}{\partial x}N-\dfrac{\partial u}{\partial y}M \\
          \\
        $
        Assume u only HAS "y", then: \\
        \\
        $
          u\left[\dfrac{\partial M}{\partial y}-\dfrac{\partial N}{\partial x}\right]=-\dfrac{du}{dy}M \\
          \\
          \dfrac{du}{dy}M=u\left[\dfrac{\partial N}{\partial x}-\dfrac{\partial M}{\partial y}\right] \\
          \\
          \bigints \dfrac{du}{u}=\bigints \dfrac{\left[\dfrac{\partial N}{\partial x}-\dfrac{\partial M}{\partial y}\right]}{M}dy \\
          \\
          ln u =\bigints \dfrac{\left[\dfrac{\partial N}{\partial x}-\dfrac{\partial M}{\partial y}\right]}{M}dy \\
          \\
          u=e^{\bigints \dfrac{\left[\dfrac{\partial N}{\partial x}-\dfrac{\partial M}{\partial y}\right]}{M}dy} \\
          \\
          \Longrightarrow u=e^{\bigints \dfrac{N_x-M_y}{M}dy} \\
        $
        For $u$ with no $"x"$ and only $"y"$.
      }
      

    \item Show that the following equations have the form $dU(x,y)=0$ (i.e., the left side of the equation is an exact differential). For each equation, find the general solution and the specific solution that satisfies the initial condition given.
    \begin{equation*}
      a. ~~ (2y+3x)dy+(3x^2+3y)dx=0, ~~~ y(0)=1
    \end{equation*}
    \begin{equation*}
      b. ~~ (3x^2y^2+e^y)dy+2(xy^3+1)dx=0, ~~~ y(0)=2
    \end{equation*}
    \begin{equation*}
      c. ~~ (xe^y-1)dy+e^ydx=0 ~~~ y(0)=2
    \end{equation*}

      \textcolor{hwColor}{ 
        $a. ~~~ (3x^2+3y)dx+(2y+3x)dy=0 \rightarrow dF=Mdx+Ndy=0$ \\
        $
        \begin{cases}
          \dfrac{\partial F}{\partial x}=F_x=M=3x^2+3y \rightarrow \dfrac{\partial M}{\partial y}=3 \\
          \\
          \dfrac{\partial F}{\partial y}=F_y=N=2y+3x \rightarrow \dfrac{\partial N}{\partial x}=3
        \end{cases}
        $ \\
        $\dfrac{\partial M}{\partial y}=\dfrac{\partial N}{\partial x}=3 \Longrightarrow$ We have an exact differential equation. \\
        $F(x,y)=\bigints Mdx+g(y)=\bigints (3x^2+3y)dx+g(y)=x^3+3xy+g(y)$ \\
        $\dfrac{\partial F}{\partial y}=0+3x+g^{\prime}(y)=2y+3x \rightarrow g^{\prime}(y)=2y \rightarrow g(y)=\bigints 2ydy=y^2$ \\
        $F(x,y)=x^3+3xy+y^2=C ~~~~ y(0)=1 \rightarrow 0^3+3(0)(1)+1^2=C \Longrightarrow C=1$ \\
        \\
        $F(x,y)=x^3+3xy+y^2-1=0$
      }

      \textcolor{hwColor}{  
        \rule{15cm}{0.4pt}  
      }

      \textcolor{hwColor}{ 
        $b. ~~~ 2(xy^3+1)dx+(3x^2y^2+e^y)dy=0$ \\
        $
        \begin{cases}
          \dfrac{\partial F}{\partial x}=F_x=M=2(xy^3+1) \rightarrow \dfrac{\partial M}{\partial y}=6xy^2 \\
          \\
          \dfrac{\partial F}{\partial y}=F_y=N=3x^2y^2+e^y \rightarrow \dfrac{\partial N}{\partial x}=6xy^2
        \end{cases}
        $ \\
        $\dfrac{\partial M}{\partial y}=\dfrac{\partial N}{\partial x}=6xy^2 \Longrightarrow$ We have an exact differential equation. \\
        $F(x,y)=\bigints Mdx+g(y)=\bigints 2(xy^3+1) dx+g(y)=x^2y^3+2x+g(y)$ \\
        $\dfrac{\partial F}{\partial y}=3x^2y^2+g^{\prime}(y)=3x^2y^2+e^y \rightarrow g(y)=\bigints e^ydy=e^y$ \\
        $F(x,y)=x^2y^3+2x+e^y=C ~~~~ y(0)=2 \rightarrow 0^2(2)^3+2(0)+e^2=C \Longrightarrow C=e^2$ \\
        \\
        $F(x,y)=x^2y^3+2x+e^y-e^2=0$
      }

      \textcolor{hwColor}{  
        \rule{15cm}{0.4pt}  
      }

      \textcolor{hwColor}{ 
        $c. ~~~ e^ydx+(xe^y-1)dy=0 $ \\
        $
        \begin{cases}
          \dfrac{\partial F}{\partial x}=F_x=M=e^y \rightarrow \dfrac{\partial M}{\partial y}=e^y \\
          \\
          \dfrac{\partial F}{\partial y}=F_y=N=xe^y-1 \rightarrow \dfrac{\partial N}{\partial x}=e^y
        \end{cases}
        $ \\
        $\dfrac{\partial M}{\partial y}=\dfrac{\partial N}{\partial x}=e^y \Longrightarrow$ We have an exact differential equation. \\
        $F(x,y)=\bigints Mdx+g(y)=\bigints e^y dx+g(y)=xe^y+g(y)$ \\
        $\dfrac{\partial F}{\partial y}=xe^y-1=xe^y+g^{\prime}(y) \rightarrow g^{\prime}(y)=-1 \rightarrow g(y)=-\bigints dy=-y$ \\
        $F(x,y)=xe^y-y=C ~~~~ y(0)=2 \rightarrow 0e^2-2=C \Longrightarrow C=-2$ \\
        \\
        $F(x,y)=xe^y-y+2=0$
      }
    
    \item Solve each of the following differential equations using the integrating factor method (regardless of whether they are solvable by other means) to render them exact:
    \begin{equation*}
      a. ~~ 2xdy+y(1+x)dx=0
    \end{equation*}
    \begin{equation*}
      b. ~~ cos(x-y)y^{\prime}-cos(x-y)-sin(x-y)=0
    \end{equation*}
    \begin{equation*}
      c. ~~  (y+e^x)y^{\prime}+2ye^x+\dfrac{y^2}{2}=0
    \end{equation*}
    \begin{equation*}
      d. ~~ (y+x)y^{\prime}+\dfrac{y-1}{x}+2+e^{-y}=0
    \end{equation*}
  \end{enumerate}

\end{document}