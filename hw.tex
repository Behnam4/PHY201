\documentclass[fleqn]{article}
\oddsidemargin 0.0in
\textwidth 6.0in
\thispagestyle{empty}
\usepackage{import}
\usepackage{amsmath}
\usepackage{bigints} 
\usepackage{graphicx}
\usepackage[english]{babel}
\usepackage[utf8x]{inputenc}
\usepackage{float}
\usepackage[colorinlistoftodos]{todonotes}

\definecolor{hwColor}{HTML}{AD53BA}

\begin{document}

  \begin{titlepage}

    \newcommand{\HRule}{\rule{\linewidth}{0.5mm}} % Defines a new command for the horizontal lines, change thickness here

    \center % Center everything on the page



    \textsc{\LARGE Arizona State University}\\[1.5cm] % Name of your university/college

    \textsc{\LARGE Mathematical Methods For Physics I }\\[1.5cm] % Major heading such as course name


    \begin{figure}
      \includegraphics[width=\linewidth]{asu.png}
    \end{figure}


    \HRule \\[0.4cm]
    { \huge \bfseries Homework 13}\\[0.4cm] 
    \HRule \\[1.5cm]

    \textbf{Behnam Amiri}

    \bigbreak

    \textbf{Prof: Cecilia Lunardini}

    \bigbreak


    \textbf{{\large \today}\\[2cm]}

    \vfill % Fill the rest of the page with whitespace

  \end{titlepage}

  \begin{enumerate}
    \item Use the method of the variation of the parameters to find the general solution the following second order linear ODEs (here primes indicate derivatives with respect to $x$): 
      \begin{enumerate}
      \item $y'' + y = x \sin x$

      \item $y'' + y = x e^{-2x }$

      \item $y'' + y = e^{-x} \sin x$ 
      \end{enumerate}

      \textcolor{hwColor}{
        (a): \\ 
        $ 
          y''+y=0 \rightarrow y''+0y'+y=0 \\
          \\
          \lambda^2+0\lambda+1=0 \Rightarrow \begin{cases}
            \lambda_1= 0+i=\alpha+i\beta \\
            \lambda_2= 0-i=\alpha-i\beta \\
          \end{cases} \longrightarrow \alpha=0 ~~~~~~ \beta=1 \\
          \\
          y_c=e^{\alpha x}\left[c_1 cos(x)+c_2 sin(x)\right]=c_1 cos(x)+c_2 sin(x)=u_1y_1+u_2y_2 \\
          \\
          \Longrightarrow \begin{cases}
            u_1=c_1 \\
            y_1=cos(x) \\
            u_2=c_2 \\
            y_2=sin(x) \\
          \end{cases} \\
          \\
          y_p=u_1y_1+u_2y_2 \\
          \\
        $
        Condition:
        $
          u^{\prime}_1y_1+u^{\prime}_2y_2=0 \rightarrow u^{\prime}_1cos(x)+u^{\prime}_2sin(x)=0 \\
          \\
          y_p=u_1cos(x)+u_2sin(x) ~~~~~ y^{\prime}_p=u^{\prime}_1cos(x)-u_1sin(x)+u^{\prime}_2sin(x)+u_2cos(x) \\
          \\
          y^{\prime}_p=-u_1sin(x)+u_2cos(x) \\
          \\
          y''_p=-u^{\prime}_1sin(x)-u_1cos(x)+u^{\prime}_2cos(x)-u_2sin(x) \\
          \\
          y''+y=xsin(x) \rightarrow y''_p+y_p=-u^{\prime}_1sin(x)-u_1cos(x)+u^{\prime}_2cos(x)-u_2sin(x)+u_1cos(x)+u_2sin(x) \\
          \\
          y''_p+y_p=-u^{\prime}_1sin(x)+u^{\prime}_2cos(x)=xsin(x) \\
          \\
        $
      }

      \textcolor{hwColor}{
        $
          \begin{cases}
            u^{\prime}_1cos(x)+u^{\prime}_2sin(x)=0 \\
            \\
            -u^{\prime}_1sin(x)+u^{\prime}_2cos(x)=xsin(x) \\
          \end{cases} \rightarrow \begin{pmatrix}
            cos(x) & sin(x) \\
            -sin(x) & cos(x) \\
          \end{pmatrix}.\begin{pmatrix}
            u^{\prime}_1 \\
            u^{\prime}_2
          \end{pmatrix}=\begin{pmatrix}
            0 \\
            xsin(x)
          \end{pmatrix} \\
          \\
          \\
          u^{\prime}_1=\dfrac{\begin{vmatrix}
            0 & sin(x) \\
            xsin(x) & cos(x)
          \end{vmatrix}}{cos^2(x)+sin^2(x)}=-xsin^2(x) ~~~~~~~ u^{\prime}_2=\dfrac{\begin{vmatrix}
            cos(x) & 0 \\
            -sin(x) & xsin(x)
          \end{vmatrix}}{cos^2(x)+sin^2(x)}=xsin(x)cos(x) \\
          \\
          \\
          u_1=\bigints -xsin^2(x)dx=-\left[-xcos(x)+\bigints cos(x)dx\right]=xcos(x)-sin(x) \\
          \\
          u_2=\bigints xsin(x)cos(x) dx=\dfrac{1}{2}\bigints x sin(2x)dx=\dfrac{1}{2}\left[-x\dfrac{1}{2} cos(2x)+\bigints \dfrac{1}{2} cos(2x)dx\right] \\
          = \dfrac{1}{4}\left[-x cos(2x)+\dfrac{1}{2}sin(2x)\right] \\
          \\
          \\
          y_p=xcos(x)-sin(x)cos(x)+\dfrac{1}{4}\left[-x cos(2x)+\dfrac{1}{2}sin(2x)\right]sin(x) \\
          \\
          \\
          y=y_c+y_p \\
          \\
          \Longrightarrow y=c_1 cos(x)+c_2 sin(x)+xcos(x)-sin(x)cos(x)+\dfrac{1}{4}\left[-x cos(2x)+\dfrac{1}{2}sin(2x)\right]sin(x)
        $
      }

    \item Use the series solution method (method of Frobenius) to solve the equation $y'' - k^2 y =0$ (where $k$ is a real constant).  Illustrate the connection with some known hyperbolic functions. [Hint: this exercise is very similar to the example shown in class; feel free to use that example for inspiration! ]
    
    \item  For each of the ODEs below, one solution is given. Using the method of the Wronskian, find a second, linearly independent, solution and write the general solution of the equation. 
      \begin{enumerate}
      \item $y'' + 4 y' + 3 y =0 $, with $y_1= e^{-x}$

      \item $(1-x^2) y'' - 2 x y' + 2y=0 $, with $y_1= x $

      \item  $x^2 y'' + x y' + (x^2 - \frac{1}{4})y=0 $, with $y_1=\sin x/\sqrt{x} $
      \end{enumerate}


    \item  Consider the one-dimensional wave equation, as written in the textbook (top of page 677). (i) Prove, by directly substituting  into the equation, that any generic function $f(k)$, where the argument $k$ is $k=x+ct$, is a solution of the equation. (ii) do the same proof for $k=x-ct$. 








  \end{enumerate}

\end{document}
