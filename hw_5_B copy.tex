\documentclass{article}
\oddsidemargin 0.0in
\textwidth 6.0in
\thispagestyle{empty}
\usepackage{import}
\usepackage{amsmath}
\usepackage{graphicx}

\begin{document}

%useful notation: \mathbf{A}=\mathbf{\hat{x}}+2\mathbf{\hat{y}-}2\mathbf{\hat{z}, \quad \quad B}=3\mathbf{\hat{x}}+\mathbf{\hat{y}+}2\mathbf{\hat{z}, \quad \quad C}=4\mathbf{\hat{x}}-\mathbf{\hat{y}+\hat{z}.}

%matrix template: A = 
%\begin{pmatrix}
%1 & 2 & 3 \\
%4 & 5 & 6 \\
%7 & 8 & 9
%\end{pmatrix}

{\bf Homework 5 - Part B}\\

Homework assigned on a given day is always due on the following Tuesday at noon. Electronic submission via Canvas is required. Single file, .pdf or .jpeg only.  A scannerized copy of a handwritten file is acceptable. Typeset solutions are welcome as well. The student is responsible for the readability of the file. 

\begin{enumerate}

\item  Consider the two matrices:
$$A=\begin{pmatrix}
2 & 5 & -1 \\
-3 & 4 & 2 \\
1 & 7 & 3
\end{pmatrix} \hskip 1truecm B=\begin{pmatrix}
5 & -5 & 3 \\
1 & 4 & 3 \\
-4 & 2 & 1
\end{pmatrix}$$ 
\begin{enumerate}
\item Compute their products  (i.e., $AB$ and $BA$)

\item  Compute their traces

\item  Compute their transposed matrices, i.e., $A^T$ and $B^T$. 
\end{enumerate}


\item Form as many products as you can from pairs of the following five matrices or their transposes ~~~( {\bf clarification}: examine all the possible pairs, and determine those for which the product is possible. How many are they?  Then, calculate 5 products of your choice. All homework calculations should be done by hand.): 
\begin{figure}[htbp]
\begin{center}
\includegraphics[width=0.7\textwidth]{tutorialVAMmatrices}
%\caption{default}
%\label{default}
\end{center}
\end{figure}


\item find the expressions of the well known Pauli matrices, and compute their Hermitian conjugates. 


\end{enumerate}

\end{document}
