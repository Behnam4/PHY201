\documentclass[fleqn]{article} 
\oddsidemargin 0.0in 
\textwidth 6.0in 
\thispagestyle{empty} 
\usepackage{import} 
\usepackage{amsmath} 
\usepackage{bigints} 
\usepackage{graphicx} 
\usepackage[english]{babel} 
\usepackage[utf8x]{inputenc} 
\usepackage{float} 
\usepackage[colorinlistoftodos]{todonotes} 
\usepackage{mathtools} 
\usepackage[thinc]{esdiff} 

\definecolor{hwColor}{HTML}{AD53BA}

\begin{document}

  \begin{titlepage}

    \newcommand{\HRule}{\rule{\linewidth}{0.5mm}} % Defines a new command for the horizontal lines, change thickness here

    \center % Center everything on the page
    


    \textsc{\LARGE Arizona State University}\\[1.5cm] % Name of your university/college

    \textsc{\LARGE Mathematical Methods For Physics I }\\[1.5cm] % Major heading such as course name


    \begin{figure}
      \includegraphics[width=\linewidth]{asu.png}
    \end{figure}


    \HRule \\[0.4cm]
    { \huge \bfseries Homework 11}\\[0.4cm] 
    \HRule \\[1.5cm]
    
    \textbf{Behnam Amiri}

    \bigbreak

    \textbf{Prof: Cecilia Lunardini}

    \bigbreak


    \textbf{{\large \today}\\[2cm]}

    \vfill % Fill the rest of the page with whitespace

  \end{titlepage}

  \begin{enumerate}

    \item  Derive an expression analogous to eq (14.19) of the textbook for the case of isobaric equations. 
    
      \textcolor{hwColor}{
        $
          \dfrac{dy}{dx}=\dfrac{A(x,y)}{B(x,y)}, ~~~~ y=vx^m \\
          \\
          \dfrac{d(vx^m)}{dx}=x^m\dfrac{dv}{dx}+mx^{m-1}v=\dfrac{A(1, vmx^{m-1})}{B(1, vmx^{m-1})} \\
          \\
          \Longrightarrow \dfrac{A(1,v)}{B(1,v)}=mv+x\dfrac{dv}{dx}=F(v) \\
          x\dfrac{dv}{dx}=F(v)-mv \rightarrow \dfrac{dv}{dx}=\dfrac{F(v)-mv}{x} \\
          \\
          \\
          \Longrightarrow \bigints \dfrac{dv}{F(v)-mv}=\bigints \dfrac{dx}{x}
        $     
      }

    
    \item  Show that the following first order ODEs are homogeneous or isobaric, identify $m$ (see textbook sec. 14.2.6 for the definition of $m$), and solve them:
      \begin{enumerate}
        \item $2xdy+(x^2-3y)dx=0$

        \textcolor{hwColor}{
          $
            2xdy+(x^2-3y)dx=0 \\
            2xdy=3y-x^2 \\
            \dfrac{dy}{dx}-(\dfrac{3}{2x})y=-\dfrac{x}{2} \rightarrow \begin{cases}
              P(x)=-\dfrac{3}{2x} \\
              \\
              Q(x)=-\dfrac{x}{2}
            \end{cases} \\
            \\
            \\
            \rho(x)=e^{\bigints P(x)dx}=e^{\bigints -\dfrac{3}{2x}dx}=x^{-\dfrac{3}{2}}
            \\
            \\
            x^{-\dfrac{3}{2}}\dfrac{dy}{dx}-x^{-\dfrac{3}{2}}(\dfrac{3}{2x})y=-(x^{-\dfrac{3}{2}})\dfrac{x}{2} \Rightarrow x^{-\dfrac{3}{2}}\dfrac{dy}{dx}-\dfrac{3x^{-\dfrac{5}{2}}}{2}y=-\dfrac{x^{-\dfrac{1}{2}}}{2} \\
            \\
            D_x\left[x^{-\dfrac{3}{2}}.y\right]=-\dfrac{x^{-\dfrac{1}{2}}}{2} \\
            \\
            \bigints \left(D_x\left[x^{-\dfrac{3}{2}}.y\right]\right)dx=\bigints \left(-\dfrac{x^{-\dfrac{1}{2}}}{2}\right)dx \\
            \\
            x^{-\dfrac{3}{2}}.y=-x^{\dfrac{1}{2}}+C \\
            \\
            \Longrightarrow y=-x^2+Cx^{\dfrac{3}{2}}
          $
        }

        \bigbreak

        \item $x y^2 (y dx - x dy)-(2y dx + x dy)=0$

          \textcolor{hwColor}{
            $
              xy^3dx-x^2y^2dy-2ydx-xdy=0 \\
              \\
              (xy^3-2y)dx=(x^2y^2+x)dy \\
              \\
              \dfrac{dy}{dx}=\dfrac{xy^3-2y}{x^2y^2+x}
            $
          }

          \bigbreak
        
        \item $2x e^{y^2/x} dy - 3y dx=0$ (for this equation, you can leave the final integration step undone, but please simplify as much as possible up to that point)

          \textcolor{hwColor}{
            $
              m+1=-1+2m+1+m \rightarrow m=\dfrac{1}{2}
              y=vx^{\dfrac{1}{2}} \\
              \\
              2xe^{\dfrac{vx^{\dfrac{1}{2}}}{x}}\left(v-\dfrac{1}{2}+x^{\dfrac{1}{2}}\dfrac{dv}{dt}\right)-3vx^{\dfrac{1}{2}}=0 \\
              2xx^{\dfrac{1}{2}}e^{v^2}\dfrac{dv}{dx}-3vx^{\dfrac{1}{2}}+vx^{\dfrac{1}{2}}e^{v^2}=0 \\
              2xe^{v^2}\dfrac{dv}{dx}+ve^{v^2}-3v=0 \\
              \\
              2e^{v^2}xdv=-ve^{v^2}dx+3vdx=(3v-ve^{v^2})dx \\
              \\
              \Longrightarrow \dfrac{1}{2}\bigints \dfrac{dx}{x}=\bigints \dfrac{e^{v^2}}{3v-ve^{v^2}}dv
            $
          }

      \end{enumerate}
    
    
    \item  Consider the following equation (which is called a Bernoulli equation, see textbook for more information): 
    \begin{equation}
    \diff{y}{x} = x  (y - y^3)~.
    \nonumber
    \end{equation}
    \begin{enumerate}
    \item Under the assumption that $y\neq 0$, make a change of variable: $z(x)=1/y^2(x)$. Rewrite the equation in terms of $x$ and $z(x)$.

      \textcolor{hwColor}{
        $
          \diff{y}{x}=xy-xy^3 \rightarrow \diff{y}{x}-xy=-xy^3 \\
          \\
          v=y^{1-3}=y^{-2} \rightarrow v=\dfrac{1}{y^2} ~~~~ y=\dfrac{1}{\sqrt{v}} \\
          \Longrightarrow \diff{y}{x}=(-\dfrac{1}{2}v^{-\dfrac{3}{2}})\diff{v}{x} \\
          \\
          \\
          -\dfrac{1}{2}v^{-\dfrac{3}{2}}\diff{v}{x}-xv^{-\dfrac{1}{2}}=-xv^{-\dfrac{1}{2}} \\
          \\
          -2v^{\dfrac{3}{2}}\left(-\dfrac{1}{2}v^{-\dfrac{3}{2}}\diff{v}{x}-xv^{-\dfrac{1}{2}}=-xv^{-\dfrac{1}{2}}\right) \\
          \\
          \dfrac{dv}{dx}+2xv=2x ~~~~~ \rho(x)=e^{\bigints 2xdx}=e^{x^2} \\
          \\
          e^{x^2}\dfrac{dv}{dx}+e^{x^2}2xv=e^{x^2}2x \\
          \\
          D_x\left[e^{x^2}.v\right]=e^{x^2}2x \rightarrow \bigints \left(D_x\left[e^{x^2}.v\right]\right)dx=\bigints e^{x^2}2xdx \\
          \\
          e^{x^2}.v=e^{x^2}+C \Longrightarrow v=\dfrac{e^{x^2}+C}{e^{x^2}} ~~~~ v=\dfrac{1}{y^2} \\
          \\
          \dfrac{1}{y^2}=\dfrac{e^{x^2}+C}{e^{x^2}} \Longrightarrow y^2=\dfrac{e^{x^2}}{e^{x^2}+C} \\
        $
      }

      \bigbreak

      \textcolor{hwColor}{
        $
          y^2=\dfrac{1}{z(x)}=\dfrac{1}{\dfrac{e^{x^2}+C}{e^{x^2}}}
        $
      }
    
    \item Solve the equation for $z(x)$ (using any method of your choice) and recast the solution in terms of $y(x)$ and $x$.

      \textcolor{hwColor}{
        $
          z(x)=\dfrac{1}{y^2(x)}=\dfrac{1}{\dfrac{e^{x^2}}{e^{x^2}+C}}=\dfrac{e^{x^2}+C}{e^{x^2}} \\
        $
      }
    
    \item {\bf for bonus credit: } Do a check of your result by solving the equation by separation of the variables instead. 
    \end{enumerate}

    \item  For each of the following equations: (i) find the general solution; (ii) impose the given initial conditions and find the particular solution that satisfies them. (Here each dot indicates a derivative with respect to t: $\dot y = dy/dt$ , $\ddot y = d^2y/dt^2$ )
      \begin{enumerate}
        \item $\ddot y+5\dot y+6y=0; ~~~~~~ y(0)=2, ~~ \dot y(0)=0;$

        \textcolor{hwColor}{
          $
            y=c_1e^{2t}+c_2te^{2t} ~~~~~ \dot y=2c_1e^{2t}+2c_2te^{2t}+c_2e^{2t} \\
            \\
            a^2+a+2=0 \rightarrow \begin{cases}
              a=-2 \\
              a=-3 \\
            \end{cases} \\
            \\
            y(0)=2=c_1+c_2 \\
            \dot y(0)=0=-2c_1-3c_2 \\
            \Longrightarrow c_1=\dfrac{6}{5} ~~~~ c_2=\dfrac{4}{5}
            \\
            \\
            y=\dfrac{6}{5}e^{-2t}+\dfrac{4}{5}e^{-3t}
          $     
        }

        \item $\ddot y-4\dot y+4y=0; ~~~~~~ y(0)=1, ~~ \dot y(0)=1;$

          \textcolor{hwColor}{
            $
              a^2-4a+4=0 \rightarrow (a-2)(a-2)=0 \rightarrow a=2 \\
              \\
              y=c_1e^{2t}+c_2te^{2t} ~~~~~ \dot y=2c_1e^{2t}+2c_2te^{2t}+c_2e^{2t} \\
              \\
              \\
              y(0)=1=c_1e^0+0c_2e^0 \Longrightarrow c_1=1 \\
              \dot y(0)=1=2c_1+2c_2(0)+c_2 \Longrightarrow c_2=-1
              \\
              \\
              y=e^{2t}-te^{2t}
            $     
          }

        \item $\ddot y-\dot y=0; ~~~~~~ y(0)=2, ~~ \dot y(0)=-2;$

          \textcolor{hwColor}{
            $
              a^2-a=0 \rightarrow a(a-1)=0 \rightarrow \begin{cases}
                a=0 \\
                a=1
              \end{cases}\\
              \\
              y(0)=2=c_1e^0+c_2e^0 \Longrightarrow c_1=4 \\
              \dot y(0)=-2=c_2 \Longrightarrow c_2=-2 \\
              \\
              \\
              y=4e^{2t}-2te^{2t}
            $     
          }

        \item $\ddot y-4\dot y=0 ~~~~~~ y(0)=0, ~~ \dot y(0)=1;$

          \textcolor{hwColor}{
            $
              y=c_1e^{2t}+c_2te^{2t} ~~~~~ \dot y=2c_1e^{2t}+2c_2te^{2t}+c_2e^{2t} \\
              \\
              a^2-4a=0 \rightarrow \begin{cases}
                a=0 \\
                a=4 \\
              \end{cases} \\
              \\
              y(0)=0=c_1 \\
              \dot y(0)=1=2c_1+c_2 \rightarrow c_2=1 \\
              \\
              \\
              y=te^{2t}
            $     
          }

        \item $\ddot y+\dot y+2y=0 ~~~~~~ y(0)=2, ~~ \dot y(0)=0;$
        
          \textcolor{hwColor}{
            $
              y=c_1e^{2t}+c_2te^{2t} ~~~~~ \dot y=2c_1e^{2t}+2c_2te^{2t}+c_2e^{2t} \\
              \\
              a^2+a+2=0 \rightarrow \begin{cases}
                a=-\dfrac{1}{2}+i\dfrac{\sqrt{7}}{2} \\
                a=-\dfrac{1}{2}-i\dfrac{\sqrt{7}}{2} \\
              \end{cases} \\
              \\
              y(0)=0=c_1 \\
              \dot y(0)=1=2c_1+c_2 \rightarrow c_2=1 \\
              \\
              \\
            $     
          }

      \end{enumerate}
    
  \end{enumerate}

\end{document}
