\documentclass[fleqn]{article}
\oddsidemargin 0.0in
\textwidth 6.0in
\thispagestyle{empty}
\usepackage{import}
\usepackage{amsmath}
\usepackage{bigints}
\usepackage{graphicx}
\usepackage[english]{babel}
\usepackage[utf8x]{inputenc}
\usepackage{float}
\usepackage[colorinlistoftodos]{todonotes}

\definecolor{hwColor}{HTML}{AD53BA}

\begin{document}

  \begin{titlepage}

    \newcommand{\HRule}{\rule{\linewidth}{0.5mm}} % Defines a new command for the horizontal lines, change thickness here

    \center % Center everything on the page

    \textsc{\LARGE Arizona State University}\\[1.5cm] % Name of your university/college

    \textsc{\LARGE Mathematical Methods For Physics I }\\[1.5cm] % Major heading such as course name


    \begin{figure}
      \includegraphics[width=\linewidth]{asu.png}
    \end{figure}


    \HRule \\[0.4cm]
    { \huge \bfseries Homework 14}\\[0.4cm] 
    \HRule \\[1.5cm]

    \textbf{Behnam Amiri}

    \bigbreak

    \textbf{Prof: Cecilia Lunardini}

    \bigbreak


    \textbf{{\large \today}\\[2cm]}

    \vfill % Fill the rest of the page with whitespace

  \end{titlepage}

  \begin{enumerate}
    \item  Solve the problem of the vibrating string of length $L$ (covered in class) with the function
      $$f(x)=\begin{cases} \frac{4 h x}{L} &\mbox{if } 0 \leq x < L/4 \\ 
      2 h \left( 1 - \frac{2x}{L}\right) &\mbox{if }  L/4  \leq x \leq L/2 \\ 
      0 & \mbox{if } x > L/2 \end{cases} $$
      describing the initial displacement of the string at $t=0$. Plot $f(x)$. Assume that initially the string is released from rest. [Note: please do all the steps, including the separation of the variables, even though many of them have been shown in class. ]

      \textcolor{hwColor}{
        $
          \dfrac{\partial^2y}{\partial x^2}-\dfrac{1}{v^2}\dfrac{\partial^y}{\partial t^2}=0
          \\
          \\
          y(x,t)=X(x)T(t) \\
          \\
          \dfrac{\partial^2}{\partial x^2}\left(X(x)T(t)\right)-\dfrac{1}{v^2}\dfrac{\partial^2}{\partial t^2}\left(X(x)T(t)\right)=0 \\
          \\
          T(t)\dfrac{d^2X}{dx^2}-\dfrac{1}{v^2}X(x)\dfrac{d^2T}{dt^2}=0 \\
          \\
        $ 
        Now divide both sides by $\dfrac{1}{X(x)T(t)}$, so we get:
        $
          \dfrac{1}{X(x)}\dfrac{d^2X}{dx^2}-\dfrac{1}{v^2}\dfrac{1}{T(t)}\dfrac{d^2T}{dt^2}=0 \\
          \\
          \\
          \dfrac{1}{X}\dfrac{d^2x}{dx^2}=\dfrac{1}{v^2}\dfrac{1}{T}\dfrac{d^2T}{dt^2} \Rightarrow \begin{cases}
            X(x)=X \\
            T(t)=T
          \end{cases}
        $ \\
        \\
        \\
        $
          \begin{cases}
            \dfrac{d^2X}{dx^2}=-K^2X \\
            \\
            \dfrac{d^2T}{dt^2}=-(vK)^2T
          \end{cases} \Rightarrow \begin{cases}
            \dfrac{d^2X}{dx^2}+K^2X=0 \\
            \\
            \dfrac{d^2T}{dt^2}+(vK)^2T=0
          \end{cases}
        $ where K is a real number. \\
        \\
      }

      \textcolor{hwColor}{
        These two objects on each side of the equation have nothing to do with each other because one of them is a function of x only depends on x. And the other part this on this side only depends on T. 
        How can these two sides be equal? Only when they are constant. \\
        \\
        $
          \begin{cases}
            X(x)=Acos(Kx)+Bsin(Kx) \\
            T(t)=Ccos(Kvt)+Dsin(Kvt)
          \end{cases}
        $
        So for our general solution we have: \\
        \\
        \\
        $
          y(x,t)=\left[Acos(Kx)+Bsin(Kx)\right] \times \left[Ccos(Kvt)+Dsin(Kvt)\right]
        $
        \\
      }

      \textcolor{hwColor}{
        We assume that the ends of the string are fixed. \\
        $
          \begin{cases}
            y(0,t)=0 \Rightarrow (i) ~~ X(0)=0\\
            y(L,t)=0 \Rightarrow (ii) ~~ X(L)=0
          \end{cases}
          \\
          \\
          \\
          (i) ~~ X(x)=Acos(Kx)+Bsin(Kx) \\
          x(0)=Acos(0)+Bsin(0) \longrightarrow A=0 \\
          \\
          \\
          (ii) ~~ X(x)=Bsin(Kx) \\
          X(L)=0 \rightarrow Bsin(KL)=0 \Rightarrow KL=n\pi ~~
        $ where $n=1,2,3,...$ \\
      }

      \textcolor{hwColor}{
        $
          y(x,t)=\sum\limits_{n=1}^{\infty} sin(\dfrac{n\pi}{L}x) \times \left[C_ncos(K_nvt)+D_nsin(K_nvt)\right]
        $
         If $\omega_n=K_nv$ then: \\ 
        $
          \\
          y(x,t)=\sum\limits_{n=1}^{\infty} sin(K_nx)\left[C_ncos(\omega_n t)+D_nsin(\omega_n t)\right]
        $ Where $K_n=\dfrac{n\pi}{L}$
      }

      \textcolor{hwColor}{
        The initial condition is that the string is released from rest which means $y^{'}(x,0)=0$ \\
        \\
        $
          \dfrac{dy}{dt}=\dfrac{d}{dt}\left[\sum\limits_{n=1}^{\infty} sin(K_nx)\left[C_ncos(\omega_n t)+D_nsin(\omega_n t)\right]\right] \\
          \\
          \dfrac{dy}{dt}=\sum\limits_{n=1}^{\infty} \omega_n sin(K_nx)\left[-C_n sin(0)+D_n cos(0)\right]=0 \\
          \\
          \dfrac{dy}{dt}=\sum\limits_{n=1}^{\infty} D_n \omega_n sin(K_nx)=0 \rightarrow D_n=0 \\
          \\
          y(x,t)=\sum\limits_{n=1}^{\infty} C_n sin(K_nx)cos(\omega_n t) \\
          \\
          \\
          y(x,0)=\sum\limits_{n=1}^{\infty} C_n sin(K_nx)cos(0)=f(x) \\
          \\
          y(x,0)=\sum\limits_{n=1}^{\infty} C_n sin(\dfrac{n\pi}{L}x)=f(x) \\
        $
      }

      \textcolor{hwColor}{
        We can find the value of $C_n$ by using the equation 12.6 from the textbook: \\
        $
          C_n=\dfrac{2}{L} \bigints_{0}^{L} f(x) sin(\dfrac{n \pi x}{L})dx \\
          \\
          =\dfrac{2}{L} \left[\bigints_{0}^{\dfrac{L}{4}}\dfrac{4hx}{L}sin(\dfrac{n \pi x}{\dfrac{L}{4}})dx+\bigints_{\dfrac{L}{4}}^{\dfrac{L}{2}}2h(1-\dfrac{2x}{L})sin(\dfrac{n \pi x}{\dfrac{L}{4}})dx+\bigints_{\dfrac{L}{2}}^{L} 0 sin(\dfrac{n \pi x}{\dfrac{L}{4}})dx\right] \\
          \\
          =\dfrac{2}{L} \left[\bigints_{0}^{\dfrac{L}{4}}\dfrac{4hx}{L}sin(\dfrac{n \pi x}{\dfrac{L}{4}})dx+\bigints_{\dfrac{L}{4}}^{\dfrac{L}{2}}2h(1-\dfrac{2x}{L})sin(\dfrac{n \pi x}{\dfrac{L}{4}})dx\right] \\
          \\
          \\
          \\
          \bigints_{0}^{\dfrac{L}{4}}\dfrac{4hx}{L}sin(\dfrac{n \pi x}{\dfrac{L}{4}})dx=\dfrac{4h}{L} \bigints_{0}^{\dfrac{L}{4}}x sin(\dfrac{4\pi nx}{L})dx \\
          \\
          =\dfrac{4h}{L} \left[-\dfrac{Lx}{4 \pi n}cos(\dfrac{4 \pi nx}{L})+\bigints \dfrac{L}{4\pi n} cos(\dfrac{4 \pi nx}{L})dx\right] \\
          \\
          =\dfrac{4h}{L} \left[-\dfrac{Lx}{4 \pi n} cos(\dfrac{4 \pi n x}{L})+(\dfrac{L}{4 \pi n})^2 sin(\dfrac{4\pi n x}{L})\right]\Biggr|_{0}^{\dfrac{L}{4}} \\
          \\
          \\
          =\dfrac{4h}{L} \left[-\dfrac{L^2}{16 \pi n}cos(\pi n)+(\dfrac{L}{4 \pi n})^2 sin(\pi n)\right]=-\dfrac{Lh}{4 \pi n}(-1)^n \\
          \\
          \\
          \\
          \bigints_{\dfrac{L}{4}}^{\dfrac{L}{2}}2h(1-\dfrac{2x}{L})sin(\dfrac{n \pi x}{\dfrac{L}{4}})dx=2h\left[\bigints sin(\dfrac{4\pi nx}{L})dx-\bigints \dfrac{2x}{L}sin(\dfrac{4\pi nx}{L})\right]\Biggr|_{\dfrac{L}{4}}^{\dfrac{L}{2}} \\
          \\
          =2h \left[-\dfrac{L}{4 \pi n} cos(\dfrac{4 \pi nx}{L})+\dfrac{x}{2\pi n}cos(\dfrac{4 \pi nx}{L})-\dfrac{L}{8(n\pi)^2}sin(\dfrac{4\pi nx}{L})\right]\Biggr|_{\dfrac{L}{4}}^{\dfrac{L}{2}} \\
          \\
          =2h \left[-\dfrac{L}{4 \pi n}+\dfrac{L}{4 \pi n}+\dfrac{L}{4 \pi n}(-1)^n-\dfrac{1}{8 \pi n}(-1)^n\right]=2h \left[\dfrac{L}{4 \pi n}(-1)^n-\dfrac{1}{8 \pi n}(-1)^n\right] \\
          \\
          =\dfrac{h(-1)^n (2L-1)}{4 \pi n} \\
          \\
          \\
          \\
          \\
          C_n=\dfrac{2}{L} \left[-\dfrac{Lh}{4 \pi n}(-1)^n+\dfrac{h(-1)^n (2L-1)}{4 \pi n}\right] \\
          \\
          \\
          \Longrightarrow C_n=\dfrac{h(-1)^n(L-2L^2)}{4 \pi n} \\
          \\
          \\
          \Longrightarrow y(x,0)=\sum\limits_{n=1}^{\infty} \dfrac{h(-1)^n(L-2L^2)}{4 \pi n} sin(\dfrac{n\pi}{L}x)=f(x) \\
          \\
          \\
          \\
          \\
          \\
          \\
        $
      }

    \item  

      \begin{enumerate}
      \item Derive the differential equation for a  vibrating string in presence of gravity. Assume that the string has constant mass per unit length, $\mu$, and that at rest the string lies horizontally (like in a piano); assume small displacements, purely in the transverse direction (like in the case discussed in class).  

      \item check if the new equation can still be solved by separation of the variables (it is not necessary to find the solution).  [ Optional: reflect on your result. Would a musical instrument sound the same on Earth and on a future human colony on the Moon?  ] 

      \end{enumerate}

    \item Generalize the problem of the vibrating string to the case of a vibrating rectangular drum of sides $a$ and $b$. Consider the 2-dimensional wave equation describing the (transverse) displacement, $U(x,y,t)$, of the drum's membrane:
    $$
    \frac{\partial^2 U}{\partial x^2} + \frac{\partial^2 U}{\partial y^2}  - \frac{1}{v^2}\frac{\partial^2 U}{\partial t^2}=0
    $$
    and start to solve it using the method of the separation of the variables. Proceed until you obtain three separate ordinary differential equations. 
    
    \item {\bf Bonus: } continue the previous problem: discuss the signs of the linking constants and show that they must take discrete values if we impose that the edges of the drum are fixed at all times. 
  \end{enumerate}

\end{document}
