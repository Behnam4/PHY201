\documentclass[fleqn]{article}
\oddsidemargin 0.0in
\textwidth 6.0in
\thispagestyle{empty}
\usepackage{import}
\usepackage{amsmath}
\usepackage{graphicx}
\usepackage[english]{babel}
\usepackage[utf8x]{inputenc}
\usepackage{float}
\usepackage[colorinlistoftodos]{todonotes}

\begin{document}

\begin{titlepage}

\newcommand{\HRule}{\rule{\linewidth}{0.5mm}} % Defines a new command for the horizontal lines, change thickness here

\center % Center everything on the page
 


\textsc{\LARGE Arizona State University}\\[1.5cm] % Name of your university/college

\textsc{\Large Mathematical Methods For Physics I }\\[0.5cm] % Major heading such as course name
\textsc{\large PHY 201}\\[0.5cm] % Minor heading such as course title



\begin{figure}
  \includegraphics[width=\linewidth]{asu.png}
\end{figure}



\HRule \\[0.4cm]
{ \huge \bfseries Homework 5}\\[0.4cm] 
\HRule \\[1.5cm]
 
\textbf{Behnam Amiri}

\bigbreak

\textbf{Prof: Cecilia Lunardini}

\bigbreak


\textbf{{\large \today}\\[2cm]}

\vfill % Fill the rest of the page with whitespace

\end{titlepage}


\begin{enumerate}


  \item Consider the vectors: $\mathbf{a}=( \mathbf{\hat{x}} - \mathbf{\hat{y}} ) / \sqrt{2} $, $\mathbf{b}=( \mathbf{\hat{x}} + \mathbf{\hat{y}} + 2 \mathbf{\hat{z}} ) / \sqrt{6}$ and $\mathbf{c}=(\mathbf{\hat{x}}+\mathbf{\hat{y}}-\mathbf{\hat{z}})/\sqrt{3}$. 
  % . 
  \begin{enumerate}
  \item Prove that they are a basis for a three dimensional cartesian space, and check if they are an \emph{orthogonal} basis. 
  \\
  
  So to show that the set of vectors forms a basis for $R^3$ we only need to show that all possible vectors in $R^3$ can be made from some linear combination of these vectors. This is equivalent to putting these three vectors into a matrix, and showing it row reduces to the identity matrix (since the basis vectors of $R^3$ are (1,00), (0,1,0), and (0,0,1)).
  \\
  So what we need to do first is make a, b, c the columns of a $3\times3$ matrix row reduce that matrix, so that we see it becomes the identity, and thus is a basis for $R^3$
  
  \noindent
  $
  \begin{array}{lll}
    \overrightarrow{a}: \dfrac{\hat{x}-\hat{y}}{\sqrt{2}}=\dfrac{\hat{x}}{\sqrt{2}}-\dfrac{\hat{y}}{\sqrt{2}} \\
    \overrightarrow{b}: \dfrac{\hat{x}+\hat{y}+2\hat{z}}{\sqrt{6}}=\dfrac{\hat{x}}{\sqrt{6}}+\dfrac{\hat{y}}{\sqrt{6}}+\dfrac{2\hat{z}}{\sqrt{6}} \\
    \overrightarrow{c}: \dfrac{\hat{x}+\hat{y}-\hat{z}}{\sqrt{3}}=\dfrac{\hat{x}}{\sqrt{3}}+\dfrac{\hat{y}}{\sqrt{3}}-\dfrac{\hat{z}}{\sqrt{3}}
    \end{array}  
  $
  \bigbreak
  
  \noindent
  $\displaystyle
    \begin{pmatrix}
      \dfrac{1}{\sqrt{2}} & \dfrac{-1}{\sqrt{2}} & 0 \\
      \dfrac{1}{\sqrt{6}} & \dfrac{1}{\sqrt{6}} & \dfrac{2}{\sqrt{6}} \\ 
      \dfrac{1}{\sqrt{3}} & \dfrac{1}{\sqrt{3}} & \dfrac{-1}{\sqrt{3}} \\ 
    \end{pmatrix}
    =
    \begin{pmatrix}
      1 & -1 & 0\\
      1 & 1 & 2\\ 
      1 & 1 & -1\\ 
    \end{pmatrix}
    = 
    \begin{pmatrix}
      1 & -1 & 0\\
      0 & 2 & 2\\ 
      1 & 1 & -1\\ 
    \end{pmatrix}
    = 
    \begin{pmatrix}
      1 & -1 & 0\\
      0 & 2 & 2\\ 
      0 & 2 & -1\\ 
    \end{pmatrix}
    = 
    \begin{pmatrix}
      1 & -1 & 0\\
      0 & 1 & 1\\ 
      0 & 2 & -1\\ 
    \end{pmatrix}
  $
  
  \noindent
  $\displaystyle
    =
    \begin{pmatrix}
      1 & -1 & 0\\
      0 & 1 & 1\\ 
      0 & 0 & -3\\ 
    \end{pmatrix}
    =
    \begin{pmatrix}
      1 & -1 & 0\\
      0 & 1 & 1\\ 
      0 & 0 & 1\\ 
    \end{pmatrix}
    =
    \begin{pmatrix}
      1 & -1 & 0\\
      0 & 1 & 0\\ 
      0 & 0 & 1\\ 
    \end{pmatrix}
    =
    \begin{pmatrix}
      1 & 0 & 0\\
      0 & 1 & 0\\ 
      0 & 0 & 1\\ 
    \end{pmatrix}
    =
    I_3\times_3
  $
  
  \bigbreak
  
  To show that it is an orthogonal basis, we need to check that a, b, and c are perpendicular to each other, thus cross product of one with the other two will give 0.
  
  \bigbreak
  
  $
  \overrightarrow{a}\times(\overrightarrow{b}\times\overrightarrow{c})=0
  $
  
  $
  \overrightarrow{b}\times\overrightarrow{c}
  =
  \left|
  \begin{matrix}
    \hat{x} & \hat{y} & \hat{z} \\
    \dfrac{1}{\sqrt{6}} & \dfrac{1}{\sqrt{6}} & \dfrac{2}{\sqrt{6}} \\ 
    \dfrac{1}{\sqrt{3}} & \dfrac{1}{\sqrt{3}} & \dfrac{-1}{\sqrt{3}} \\ 
  \end{matrix}
  \right|
  =
  \left|
  \begin{matrix}
    \hat{x} & \hat{y} & \hat{z} \\
    1 & 1 & 2 \\ 
    1 & 1 & -1 \\ 
  \end{matrix}
  \right|
  =
  \hat{x}(-1-2)-\hat{y}(-1-2)+\hat{z}(1-1)
  =
  -3\hat{x}+3\hat{y}
  $
  
  Now $\overrightarrow{b}\times\overrightarrow{c}=\overrightarrow{V}$ vector has to be parallel with $\overrightarrow{a}$ hence, the cross product of these two vectors will give 0.
  
  $
  \overrightarrow{a}\times\overrightarrow{V}
  =
  \left|
  \begin{matrix}
    \hat{x} & \hat{y} & \hat{z} \\
    \dfrac{1}{\sqrt{2}} & \dfrac{-1}{\sqrt{2}} & 0  \\ 
    -3 & 3 & 0 \\ 
  \end{matrix}
  \right|
  =
  \hat{x}(0-0)-\hat{y}(0-0)+\hat{z}(0-0)
  =
  0
  $
  
  \item Express the vectors $\mathbf{\hat{x}}, \mathbf{\hat{y}}, \mathbf{\hat{z}}$ in the basis of the vectors $\mathbf{a}, \mathbf{b}, \mathbf{c}$. 
  
  $
  \begin{array}{lll}
    \overrightarrow{a}= \dfrac{1}{\sqrt{2}}(\hat{x}-\hat{y}) \\
    \overrightarrow{b}= \dfrac{1}{\sqrt{6}}(\hat{x}+\hat{y}+2\hat{z}) \\
    \overrightarrow{c}= \dfrac{1}{\sqrt{3}}(\hat{x}+\hat{y}-\hat{z})
    \end{array}  
  $
  
  $
  \begin{array}{lll}
    \hat{x}= (1,0,0) \\
    \hat{y}= (0,1,0) \\
    \hat{z}= (0,0,1)
    \end{array}  
  $
  
  From the above data we can create an augmented matrix and row-reduce.
  
  
  \makeatletter
  \renewcommand*\env@matrix[1][*\c@MaxMatrixCols c]{
    \hskip -\arraycolsep
    \let\@ifnextchar\new@ifnextchar
    \array{#1}}
  \makeatother
  \begin{equation}
    \begin{bmatrix}[ccc|c]
      \dfrac{1}{\sqrt{2}} & \dfrac{-1}{\sqrt{2}} & 0 & \overrightarrow{a} \\
      \dfrac{1}{\sqrt{6}} & \dfrac{1}{\sqrt{6}} & \dfrac{2}{\sqrt{6}} & \overrightarrow{b} \\
      \dfrac{1}{\sqrt{3}} & \dfrac{1}{\sqrt{3}} & \dfrac{-1}{\sqrt{3}} & \overrightarrow{c} \\
    \end{bmatrix}
    =
    \begin{bmatrix}[ccc|c]
      1 & 0 & 0 & \dfrac{1}{6}(3\sqrt{2}\overrightarrow{a}+\sqrt{6}\overrightarrow{b}+2\sqrt{3}\overrightarrow{c}) \\
      0 & 1 & 0 & \dfrac{1}{6}(-3\sqrt{2}\overrightarrow{a}+\sqrt{6}\overrightarrow{b}+2\sqrt{3}\overrightarrow{c}) \\
      0 & 0 & 1 & \dfrac{1}{3}(\sqrt{6}\overrightarrow{b}-\sqrt{3}\overrightarrow{c}) \\
    \end{bmatrix}
  \end{equation}
  
  $\Rightarrow$ 
  $
  \begin{array}{lll}
    \hat{x}= \dfrac{1}{6}(3\sqrt{2}\overrightarrow{a}+\sqrt{6}\overrightarrow{b}+2\sqrt{3}\overrightarrow{c}) \\
    \hat{y}= \dfrac{1}{6}(-3\sqrt{2}\overrightarrow{a}+\sqrt{6}\overrightarrow{b}+2\sqrt{3}\overrightarrow{c}) \\
    \hat{z}= \dfrac{1}{3}(\sqrt{6}\overrightarrow{b}-\sqrt{3}\overrightarrow{c})
  \end{array}  
  $
  
  
  \bigbreak
  
  \item Consider now the set of vectors $\mathbf{A}=\mathbf{\hat{x}}-\mathbf{\hat{y}}$; $\mathbf{B}=\mathbf{\hat{x}}+\mathbf{\hat{z}}$; $\mathbf{C}=\mathbf{\hat{y}}-2\mathbf{\hat{z}}$, and express them in the basis of the vectors $\mathbf{a}, \mathbf{b}, \mathbf{c}$. 
  
  $
  \begin{array}{lll}
    \hat{x}= \dfrac{1}{6}(3\sqrt{2}\overrightarrow{a}+\sqrt{6}\overrightarrow{b}+2\sqrt{3}\overrightarrow{c}) \\
    \hat{y}= \dfrac{1}{6}(-3\sqrt{2}\overrightarrow{a}+\sqrt{6}\overrightarrow{b}+2\sqrt{3}\overrightarrow{c}) \\
    \hat{z}= \dfrac{1}{3}(\sqrt{6}\overrightarrow{b}-\sqrt{3}\overrightarrow{c})
    \end{array}  
  $
  
  
  $\Rightarrow$ 
  $
  \begin{array}{lll}
    A = \hat{x}-\hat{y}=(\dfrac{1}{6}(3\sqrt{2}\overrightarrow{a}+\sqrt{6}\overrightarrow{b}+2\sqrt{3}\overrightarrow{c}))-(\dfrac{1}{6}(-3\sqrt{2}\overrightarrow{a}+\sqrt{6}\overrightarrow{b}+2\sqrt{3}\overrightarrow{c})) \\
    B = \hat{x}+\hat{z}= (\dfrac{1}{6}(3\sqrt{2}\overrightarrow{a}+\sqrt{6}\overrightarrow{b}+2\sqrt{3}\overrightarrow{c}))+(\dfrac{1}{3}(\sqrt{6}\overrightarrow{b}-\sqrt{3}\overrightarrow{c}))\\
    C = \hat{y}-2\hat{z}=(\dfrac{1}{6}(-3\sqrt{2}\overrightarrow{a}+\sqrt{6}\overrightarrow{b}+2\sqrt{3}\overrightarrow{c}))-2(\dfrac{1}{3}(\sqrt{6}\overrightarrow{b}-\sqrt{3}\overrightarrow{c}))
  \end{array}  
  $
  
  \bigbreak
  
  $\Rightarrow$ 
  $
  \begin{array}{lll}
    A = \sqrt{2}\overrightarrow{a} \\
    B = \dfrac{\sqrt{2}}{2}(\overrightarrow{a}+\sqrt{3}\overrightarrow{b}) \\
    C = \dfrac{-\sqrt{2}}{2}\overrightarrow{a}-\dfrac{\sqrt{6}}{2}\overrightarrow{b}+\sqrt{3}\overrightarrow{c} 
  \end{array}  
  $
  
  \end{enumerate}
  
  
  \item Same as the previous exercise, with $\mathbf{a}=\frac{1}{\sqrt{6}} ( \mathbf{\hat{x}} + \mathbf{\hat{y}} + 2 \mathbf{\hat{z}} ) $, $\mathbf{b}=\frac{1}{\sqrt{30}} (\mathbf{\hat{x}} - 5\mathbf{\hat{y}} + 2\mathbf{\hat{z}})$ and $\mathbf{c}= \frac{1}{\sqrt{5}} ( 2 \mathbf{\hat{x} }- \mathbf{\hat{z} })$, and 
  $A=\mathbf{\hat{x}}+\mathbf{\hat{y}}+2\mathbf{\hat{z}}; B=\mathbf{\hat{x}}-\mathbf{\hat{y}}+2\mathbf{\hat{z}}; C=\mathbf{\hat{y}}-\mathbf{\hat{z}}$. 
  
  \noindent
  $
  \begin{array}{lll}
    \overrightarrow{a}: \dfrac{1}{\sqrt{6}}(\hat{x}+\hat{y}+2\hat{z})=\dfrac{\hat{x}}{\sqrt{6}}+\dfrac{\hat{y}}{\sqrt{6}}+\dfrac{2\hat{z}}{\sqrt{6}} \\
    \overrightarrow{b}: \dfrac{1}{\sqrt{30}}(\hat{x}-5\hat{y}+2\hat{z})=\dfrac{\hat{x}}{\sqrt{30}}-\dfrac{5\hat{y}}{\sqrt{30}}+\dfrac{2\hat{z}}{\sqrt{30}} \\
    \overrightarrow{c}: \dfrac{1}{\sqrt{5}}(2\hat{x}-\hat{z})=\dfrac{2\hat{x}}{\sqrt{5}}-\dfrac{\hat{z}}{\sqrt{5}}
    \end{array}  
  $
  \bigbreak
  
  \noindent
  $\displaystyle
    \begin{pmatrix}
      \dfrac{1}{\sqrt{6}} & \dfrac{1}{\sqrt{6}} & \dfrac{2}{\sqrt{6}}  \\
      \dfrac{1}{\sqrt{30}} & \dfrac{-5}{\sqrt{30}} & \dfrac{2}{\sqrt{30}} \\ 
      \dfrac{2}{\sqrt{5}} & 0 & \dfrac{-1}{\sqrt{5}} \\ 
    \end{pmatrix}
    =
    \begin{pmatrix}
      1 & 1 & 2\\
      1 & -5 & 2 \\ 
      2 & 0 & -1 \\ 
    \end{pmatrix}
    = 
    \begin{pmatrix}
      1 & 1 & 2\\
      2 & 0 & -1\\ 
      1 & -5 & 2\\ 
    \end{pmatrix}
    = 
    \begin{pmatrix}
      1 & 0 & 0\\
      0 & 1 & 0\\ 
      0 & 0 & 1\\ 
    \end{pmatrix}
    =
    I_3\times_3
  $
  
  \bigbreak
  
  
  $
  \overrightarrow{a}\times(\overrightarrow{b}\times\overrightarrow{c})=0
  $
  
  $
  \overrightarrow{b}\times\overrightarrow{c}
  =
  \left|
  \begin{matrix}
    \hat{x} & \hat{y} & \hat{z} \\
    \dfrac{1}{\sqrt{30}} & \dfrac{-5}{\sqrt{30}} & \dfrac{2}{\sqrt{30}} \\ 
    \dfrac{2}{\sqrt{5}} & 0 & \dfrac{-1}{\sqrt{5}} \\ 
  \end{matrix}
  \right|
  =
  \left|
  \begin{matrix}
    \hat{x} & \hat{y} & \hat{z} \\
    1 & -5 & 2 \\ 
    2 & 0 & -1 \\ 
  \end{matrix}
  \right|
  =
  \hat{x}(5-0)-\hat{y}(-1-4)+\hat{z}(0-(-10))
  $
  
  $
  =
  5\hat{x}+5\hat{y}+10\hat{z}
  =\overrightarrow{W}
  $
  
  
  $
  \overrightarrow{a}\times\overrightarrow{W}
  =
  \left|
  \begin{matrix}
    \hat{x} & \hat{y} & \hat{z} \\
    \dfrac{1}{\sqrt{6}} & \dfrac{1}{\sqrt{6}} & \dfrac{2}{\sqrt{6}}  \\ 
    5 & 5 & 10 \\ 
  \end{matrix}
  \right|
  =
  \hat{x}(10-10)-\hat{y}(10-10)+\hat{z}(5-5)
  =
  0
  $
  
  Showing the vectors $\mathbf{\hat{x}}, \mathbf{\hat{y}}, \mathbf{\hat{z}}$ in the basis of the vectors $\mathbf{a}, \mathbf{b}, \mathbf{c}$. An augmented matrix and row-reduction:
  
  \makeatletter
  \renewcommand*\env@matrix[1][*\c@MaxMatrixCols c]{
    \hskip -\arraycolsep
    \let\@ifnextchar\new@ifnextchar
    \array{#1}}
  \makeatother
  \begin{equation}
    \begin{bmatrix}[ccc|c]
      \dfrac{1}{\sqrt{6}} & \dfrac{1}{\sqrt{6}} & \dfrac{2}{\sqrt{6}} & \overrightarrow{a} \\
      \dfrac{1}{\sqrt{30}} & \dfrac{-5}{\sqrt{30}} & \dfrac{2}{\sqrt{30}} & \overrightarrow{b} \\
      \dfrac{2}{\sqrt{5}} & 0 & \dfrac{-1}{\sqrt{5}}  & \overrightarrow{c} \\
    \end{bmatrix}
    =
    \begin{bmatrix}[ccc|c]
      1 & 0 & 0 & \dfrac{\overrightarrow{a}}{6}+\dfrac{\overrightarrow{b}}{30}+\dfrac{2\overrightarrow{c}}{5} \\
      0 & 1 & 0 & \dfrac{\overrightarrow{a}}{6}-\dfrac{\overrightarrow{b}}{6} \\
      0 & 0 & 1 & \dfrac{\overrightarrow{a}}{3}+\dfrac{\overrightarrow{b}}{15}-\dfrac{\overrightarrow{c}}{5} \\
    \end{bmatrix}
  \end{equation}
  
  
  $\Rightarrow$ 
  $
  \begin{array}{lll}
    \hat{x}= \dfrac{\overrightarrow{a}}{6}+\dfrac{\overrightarrow{b}}{30}+\dfrac{2\overrightarrow{c}}{5} \\
    \hat{y}= \dfrac{\overrightarrow{a}}{6}-\dfrac{\overrightarrow{b}}{6} \\
    \hat{z}= \dfrac{\overrightarrow{a}}{3}+\dfrac{\overrightarrow{b}}{15}-\dfrac{\overrightarrow{c}}{5}
  \end{array}  
  $
  
  
  $\Rightarrow$ 
  $
  \begin{array}{lll}
    A = \hat{x}+\hat{y}+2\hat{z} \\
    B = \hat{x}-\hat{y}+2\hat{z}\\
    C = \hat{y}-\hat{z}
  \end{array}  
  $
  
  \bigbreak
  
  $\Rightarrow$ 
  $
  \begin{array}{lll}
    A = \dfrac{2\overrightarrow{a}}{3}-\dfrac{\overrightarrow{b}}{15}+\dfrac{\overrightarrow{c}}{5} \\
    B = \dfrac{2\overrightarrow{a}}{3}+\dfrac{\overrightarrow{b}}{3} \\
    C = \dfrac{-\overrightarrow{a}}{6}-\dfrac{7\overrightarrow{b}}{30}+\dfrac{\overrightarrow{c}}{5}
  \end{array}  
  $
  
  
  \item Mini-quiz: which of these sets of vectors is not a basis for a three dimensional cartesian space? 
  \begin{itemize}
  \item $\{ \mathbf{\hat{x}}, \mathbf{\hat{x}}+3\mathbf{\hat{y}}, 3\mathbf{\hat{x}}-\mathbf{\hat{y}} \}$
  
  \noindent
  $\displaystyle
    \begin{pmatrix}
      1 & 0 & 0 \\
      1 & 3 & 0 \\ 
      3 & -1 & 0 \\ 
    \end{pmatrix}
    =
    \begin{pmatrix}
      1 & 0 & 0 \\
      0 & 1 & 0 \\ 
      0 & 0 & 0 \\ 
    \end{pmatrix}
  $
  
  Therefore, the set of vectors does not form a basis for $R^3$.
  
  
  \item $\{ \mathbf{\hat{z}}, \mathbf{\hat{x}}+3\mathbf{\hat{y}}, 3\mathbf{\hat{x}}-\mathbf{\hat{y}}\}$
  
  \noindent
  $\displaystyle
    \begin{pmatrix}
      0 & 0 & 1 \\
      1 & 3 & 0 \\ 
      3 & -1 & 0 \\ 
    \end{pmatrix}
    =
    \begin{pmatrix}
      1 & 0 & 0 \\
      0 & 1 & 0 \\ 
      0 & 0 & 1 \\ 
    \end{pmatrix}
  $
  
  Therefore, the set of vectors forms a basis for $R^3$.
  
  \item $\{ e^\pi \mathbf{\hat{z}}, \mathbf{\hat{x}}+3\mathbf{\hat{y}}, 3\mathbf{\hat{x}}-\mathbf{\hat{y}}\}$
  
  \noindent
  $\displaystyle
    \begin{pmatrix}
      0 & 0 & e^\pi \\
      1 & 3 & 0 \\ 
      3 & -1 & 0 \\ 
    \end{pmatrix}
    =
    \begin{pmatrix}
      1 & 0 & 0 \\
      0 & 1 & 0 \\ 
      0 & 0 & 1 \\ 
    \end{pmatrix}
  $
  
  Therefore, the set of vectors forms a basis for $R^3$.
  \end{itemize}
  
  
  \item  Compute the sum and difference of the following two matrices:
  $$A=\begin{pmatrix}
  2 & 5 & -1 \\
  -3 & 4 & 2 \\
  1 & 7 & 3
  \end{pmatrix} \hskip 1truecm B=\begin{pmatrix}
  5 & -5 & 3 \\
  1 & 4 & 3 \\
  -4 & 2 & 1
  \end{pmatrix}$$ 
  
  Note: $A+B=B+A$
  \\
  \noindent
  $A+B=
  \begin{pmatrix}
      2 & 5 & -1 \\
      -3 & 4 & 2 \\ 
      1 & 7 & 3 \\ 
  \end{pmatrix}
  +
  \begin{pmatrix}
    5 & -5 & 3 \\
    1 & 4 & 3 \\
    -4 & 2 & 1
  \end{pmatrix}
  =
  \begin{pmatrix}
    2+5 & 5+(-5) & -1+3 \\
    -3+1 & 4+4 & 2+3 \\
    1+(-4) & 7+2 & 3+1
  \end{pmatrix}
  =
  \begin{pmatrix}
    7 & 0 & 2 \\
    -2 & 8 & 5 \\
    -3 & 9 & 4
  \end{pmatrix}
  $
  
  \noindent
  $A-B=
  \begin{pmatrix}
      2 & 5 & -1 \\
      -3 & 4 & 2 \\ 
      1 & 7 & 3 \\ 
  \end{pmatrix}
  -
  \begin{pmatrix}
    5 & -5 & 3 \\
    1 & 4 & 3 \\
    -4 & 2 & 1
  \end{pmatrix}
  =
  \begin{pmatrix}
    2-5 & 5-(-5) & -1-3 \\
    -3-1 & 4-4 & 2-3 \\
    1-(-4) & 7-2 & 3-1
  \end{pmatrix}
  =
  \begin{pmatrix}
    -3 & 10 & -4 \\
    -4 & 0 & -1 \\
    5 & 5 & 2
  \end{pmatrix}
  $
  
  \noindent
  $B-A=
  \begin{pmatrix}
    5 & -5 & 3 \\
    1 & 4 & 3 \\
    -4 & 2 & 1
  \end{pmatrix}
  -
  \begin{pmatrix}
    2 & 5 & -1 \\
    -3 & 4 & 2 \\ 
    1 & 7 & 3 \\ 
  \end{pmatrix}
  =
  \begin{pmatrix}
    5-2 & -5-5 & 3-(-1) \\
    1-(-3) & 4-4 & 3-2 \\
    -4-1 & 2-7 & 1-3
  \end{pmatrix}
  =
  \begin{pmatrix}
    3 & -10 & 4 \\
    4 & 0 & 1 \\
    -5 & -5 & -2
  \end{pmatrix}
  $
  
  \item  Consider the two matrices:
  $$A=\begin{pmatrix}
  2 & 5 & -1 \\
  -3 & 4 & 2 \\
  1 & 7 & 3
  \end{pmatrix} \hskip 1truecm B=\begin{pmatrix}
  5 & -5 & 3 \\
  1 & 4 & 3 \\
  -4 & 2 & 1
  \end{pmatrix}$$ 
  \begin{enumerate}
  \item Compute their products  (i.e., $AB$ and $BA$)
  
  \noindent
  $A.B=
  \begin{pmatrix}
    2 & 5 & -1 \\
    -3 & 4 & 2 \\
    1 & 7 & 3
  \end{pmatrix}
  .
  \begin{pmatrix}
    5 & -5 & 3 \\
    1 & 4 & 3 \\
    -4 & 2 & 1
  \end{pmatrix}
  =
  $
  
  $
  \begin{pmatrix}
    2(5)+5(1)+(-1)(-4) & 2(-5)+5(4)+(-1)(2) & 2(3)+5(3)+(-1)(1) \\
    -3(5)+4(1)+2(-4) & -3(-5)+4(4)+2(2) & -3(3)+4(3)+2(1) \\
    1(5)+7(1)+3(-4) & 1(-5)+7(4)+3(2) & 1(3)+7(3)+3(1)
  \end{pmatrix}
  =
  \begin{pmatrix}
    19 & 8 & 20 \\
    -19 & 35 & 5 \\
    0 & 29 & 27
  \end{pmatrix}
  $
  
  
  \noindent
  $B.A=
  \begin{pmatrix}
    5 & -5 & 3 \\
    1 & 4 & 3 \\
    -4 & 2 & 1
  \end{pmatrix}
  .
  \begin{pmatrix}
    2 & 5 & -1 \\
    -3 & 4 & 2 \\
    1 & 7 & 3
  \end{pmatrix}
  =
  $
  
  $
  \begin{pmatrix}
    5(2)+(-5)(-3)+3(1) & 5(5)+4(-5)+3(7) & 5(-1)+2(-5)+3(3) \\
    1(2)+4(-3)+3(1) & 1(5)+4(4)+3(7) & 1(-1)+2(4)+3(3) \\
    2(-4)+2(-3)+1(1) & 5(-4)+2(4)+1(7) & (-1)(-4)+2(2)+1(3)
  \end{pmatrix}
  =
  \begin{pmatrix}
    28 & 26 & -6 \\
    -7 & 42 & 16 \\
    -13 & -5 & 11
  \end{pmatrix}
  $
  
  
  \item  Compute their traces
  
  $
  tr(A)=2+4+3=9
  $
  
  $
  tr(B)=5+4+1=10
  $
  
  \item  Compute their transposed matrices, i.e., $A^T$ and $B^T$. 
  
  \noindent
  $A^T=
  \begin{pmatrix}
    2 & -3 & 1 \\
    5 & 4 & 7 \\
    -1 & 2 & 3
  \end{pmatrix}
  $
  
  \noindent
  $B^T=
  \begin{pmatrix}
    5 & 1 & -4 \\
    -5 & 4 & 2 \\
    3 & 3 & 1
  \end{pmatrix}
  $
  
  \end{enumerate}
  
  \item Form as many products as you can from pairs of the following five matrices or their transposes ~~~( {\bf clarification}: examine all the possible pairs, and determine those for which the product is possible. How many are they?  Then, calculate 5 products of your choice. All homework calculations should be done by hand.): 
  
  $$A=\begin{pmatrix}
    2  \\
    1  \\
    -2 
    \end{pmatrix} \hskip 1truecm M=\begin{pmatrix}
    1 & 0 & 3 \\
    0 & 1 & 2 \\
    -3 & -2 & 4
    \end{pmatrix}\hskip 1truecm P=\begin{pmatrix}
      2 & 1  \\
      1 & 2  \\
      2 & 1 
      \end{pmatrix}
    $$ 
    $$Q=\begin{pmatrix}
      3 & 2 & 1 \\
      1 & 2 & 3 \\
      \end{pmatrix} \hskip 1truecm S=\begin{pmatrix}
      3 & 4 \\
      5 & 1 \\
      \end{pmatrix}\hskip 1truecm
    $$ 

    \bigbreak
  
    \noindent
    $
    A^T=
    \begin{pmatrix}
      2 & 1 & -2
    \end{pmatrix}
    ,
    M^T=
    \begin{pmatrix}
      1 & 0 & -3 \\
      0 & 1 & -2 \\
      3 & 2 & 4
    \end{pmatrix}
    ,
    P^T=
    \begin{pmatrix}
      2 & 1 & 2 \\
      1 & 2 & 1 
    \end{pmatrix}
    $

    $
    Q^T=
    \begin{pmatrix}
      3 & 1  \\
      2 & 2  \\
      1 & 3
    \end{pmatrix}
    ,
    S^T=
    \begin{pmatrix}
      3 & 5 \\
      4 & 1 
    \end{pmatrix}
    $

    \bigbreak

    \emph{Possible}

    $A.A^T, \thinspace\thinspace\thinspace M.M^T, \thinspace\thinspace\thinspace P.P^T, \thinspace\thinspace\thinspace Q.Q^T, \thinspace\thinspace\thinspace S.S^T$

    $A.A, \thinspace\thinspace\thinspace M.M, \thinspace\thinspace\thinspace P.P, \thinspace\thinspace\thinspace Q.Q, \thinspace\thinspace\thinspace S.S, \thinspace\thinspace\thinspace Q.A$

    $M.A, \thinspace\thinspace\thinspace M.P, \thinspace\thinspace\thinspace Q.M, \thinspace\thinspace\thinspace P.Q, \thinspace\thinspace\thinspace P.S, \thinspace\thinspace\thinspace Q.P, \thinspace\thinspace\thinspace S.Q$

    \bigbreak

    \emph{Impossible}


    $A.M, \thinspace\thinspace\thinspace A.P, \thinspace\thinspace\thinspace A.Q , \thinspace\thinspace\thinspace A.S, \thinspace\thinspace\thinspace P.A, \thinspace\thinspace\thinspace S.A$

    $M.Q, \thinspace\thinspace\thinspace M.S, \thinspace\thinspace\thinspace P.M, \thinspace\thinspace\thinspace S.M, \thinspace\thinspace\thinspace S.P, \thinspace\thinspace\thinspace Q.S$

    \bigbreak

    \noindent
    $M.A=
    \begin{pmatrix}
      1 & 0 & 3 \\
      0 & 1 & 2 \\
      -3 & -2 & 4
    \end{pmatrix}
    .
    \begin{pmatrix}
      2  \\
      1  \\
      -2 
    \end{pmatrix}
    =
    \begin{pmatrix}
      1(2)+0(1)+3(-2)  \\
      0(2)+1(1)+2(-2)  \\
      2(-3)+(-2)1+4(-2) 
    \end{pmatrix}
    =
    \begin{pmatrix}
      -4  \\
      -3  \\
      -16 
    \end{pmatrix}
    $

    \noindent
    $Q.A=
    \begin{pmatrix}
      3 & 2 & 1 \\
      1 & 2 & 3
    \end{pmatrix}
    .
    \begin{pmatrix}
      2  \\
      1  \\
      -2 
    \end{pmatrix}
    =
    \begin{pmatrix}
      3(2)+2(1)+1(-2) \\
      1(2)+2(1)+3(-2)
    \end{pmatrix}
    =
    \begin{pmatrix}
      6 \\
      -2
    \end{pmatrix}
    $

    \noindent
    $A.A^T=
    \begin{pmatrix}
      2  \\
      1  \\
      -2 
    \end{pmatrix}
    .
    \begin{pmatrix}
      2 & 1 & -2
    \end{pmatrix}
    =
    \begin{pmatrix}
      2(2) & 2(1) & 2(-2) \\
      1(2) & 1(1) & 1(-2) \\
      2(-2) & 1(-2) & (-2)(-2)
    \end{pmatrix}
    =
    \begin{pmatrix}
      4 & 2 & -4 \\
      2 & 1 & -2 \\
      -4 & -2 & 4
    \end{pmatrix}
    $

    \noindent
    $P.S=
    \begin{pmatrix}
      2 & 1  \\
      1 & 2  \\
      2 & 1 
    \end{pmatrix}
    .
    \begin{pmatrix}
      3 & 4 \\
      5 & 1 
    \end{pmatrix}
    =
    \begin{pmatrix}
      2(3)+1(5) & 2(4)+1(1)  \\
      1(3)+2(5) & 1(4)+2(1)  \\
      2(3)+1(5) & 2(4)+1(1)
    \end{pmatrix}
    =
    \begin{pmatrix}
      11 & 9  \\
      13 & 6  \\
      11 & 9
    \end{pmatrix}
    $

    \noindent
    $S.S^T=
    \begin{pmatrix}
      3 & 4 \\
      5 & 1
    \end{pmatrix}
    .
    \begin{pmatrix}
      3 & 5 \\
      4 & 1
    \end{pmatrix}
    =
    \begin{pmatrix}
      3(3)+4(4) & 3(5)+4(1) \\
      5(3)+1(4) & 5(5)+1(1)
    \end{pmatrix}
    =
    \begin{pmatrix}
      25 & 19 \\
      19 & 26
    \end{pmatrix}
    $
  
  
  
  \item Find the expressions of the well known Pauli matrices, and compute their Hermitian conjugates. 
  
  $\sigma_1=\sigma_x\equiv P_1 \equiv
  \begin{pmatrix}
    0 & 1 \\
    1 & 0 
  \end{pmatrix}
  $

  $\sigma_2=\sigma_y\equiv P_2 \equiv
  \begin{pmatrix}
    0 & -i \\
    i & 0 
  \end{pmatrix}
  $

  $\sigma_3=\sigma_z\equiv P_3 \equiv
  \begin{pmatrix}
    1 & 0 \\
    0 & -1 
  \end{pmatrix}
  $

  The \emph{Hermitian} conjugate of a matrix is the \emph{transpose} of its \emph{complex conjugate}.

  \bigbreak

  $\sigma_1^*=\sigma_x^* \equiv P_1^* \equiv
  \begin{pmatrix}
    0 & 1 \\
    1 & 0 
  \end{pmatrix}
  $

  $\sigma_2^*=\sigma_y^* \equiv P_2^* \equiv
  \begin{pmatrix}
    0 & i \\
    -i & 0 
  \end{pmatrix}
  $

  $\sigma_3^*=\sigma_z^* \equiv P_3^* \equiv
  \begin{pmatrix}
    1 & 0 \\
    0 & -1 
  \end{pmatrix}
  $

  \bigbreak

  $(\sigma_1^*)^T=(\sigma_x^*)^T \equiv (P_1^*)^T \equiv P_1^\dagger \equiv
  \begin{pmatrix}
    0 & 1 \\
    1 & 0 
  \end{pmatrix}
  $

  $(\sigma_2^*)^T=(\sigma_y^*)^T \equiv (P_2^*)^T \equiv P_2^\dagger \equiv
  \begin{pmatrix}
    0 & -i \\
    i & 0 
  \end{pmatrix}
  $

  $(\sigma_3^*)^T=(\sigma_z^*)^T \equiv (P_3^*)^T \equiv P_3^\dagger \equiv
  \begin{pmatrix}
    1 & 0 \\
    0 & -1 
  \end{pmatrix}
  $

  
  
  \end{enumerate}

\end{document}
