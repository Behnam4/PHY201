\documentclass[fleqn]{article}
\oddsidemargin 0.0in
\textwidth 6.0in
\thispagestyle{empty}
\usepackage{import}
\usepackage{amsmath}
\usepackage{graphicx}
\usepackage[english]{babel}
\usepackage[utf8x]{inputenc}
\usepackage{float}
\usepackage[colorinlistoftodos]{todonotes}

\definecolor{hwColor}{HTML}{AD53BA}

\begin{document}

  \begin{titlepage}

    \newcommand{\HRule}{\rule{\linewidth}{0.5mm}} % Defines a new command for the horizontal lines, change thickness here

    \center % Center everything on the page
    


    \textsc{\LARGE Arizona State University}\\[1.5cm] % Name of your university/college

    \textsc{\LARGE Mathematical Methods For Physics I }\\[1.5cm] % Major heading such as course name


    \begin{figure}
      \includegraphics[width=\linewidth]{asu.png}
    \end{figure}


    \HRule \\[0.4cm]
    { \huge \bfseries Homework 9}\\[0.4cm] 
    \HRule \\[1.5cm]
    
    \textbf{Behnam Amiri}

    \bigbreak

    \textbf{Prof: Cecilia Lunardini}

    \bigbreak


    \textbf{{\large \today}\\[2cm]}

    \vfill % Fill the rest of the page with whitespace

  \end{titlepage}

  \begin{enumerate}
    \item Consider the function:
    \begin{equation}
      f(t)=\begin{cases}
      0~, ~~~~0< t < 1 \\
      1, ~~~~1< t< 2 
      \end{cases}
    \end{equation}
    Consider it to be periodic with period equaling the range of $t$ given above (i.e., $0<t<2$). Find the (real) Fourier series for $f(t)$. 
    
    \item Same as the previous exercise, for 
    \begin{equation}
      f(t)=\begin{cases}
      t~, ~~~~0< t < 1 \\
      t-2, ~~~~1< t< 2 
      \end{cases}
    \end{equation}


    \item Find the \emph{complex} Fourier series of the following periodic function (attention: recall that, outside the range given below, the function is repeated, see examples shown in class)
      \begin{equation}
      f(t)=\sin t~, ~~~~0< t < \pi \\
      \end{equation}

    \item Find the \emph{complex} Fourier series of the following periodic function: 
    \begin{equation}
      f(t)=\begin{cases}
      0~, ~~~~0< t < \pi \\
      \sin t, ~~~~\pi< t< 2 \pi
      \end{cases}
    \end{equation}


  \end{enumerate}
\end{document}
