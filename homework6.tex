\documentclass[fleqn]{article}
\oddsidemargin 0.0in
\textwidth 6.0in
\thispagestyle{empty}
\usepackage{import}
\usepackage{amsmath}
\usepackage{graphicx}
\usepackage[english]{babel}
\usepackage[utf8x]{inputenc}
\usepackage{float}
\usepackage[colorinlistoftodos]{todonotes}


\begin{document}

  \begin{titlepage}

    \newcommand{\HRule}{\rule{\linewidth}{0.5mm}} 

    \center
    
    \textsc{\LARGE Arizona State University}\\[1.5cm] % Name of your university/college

    \textsc{\LARGE Mathematical Methods For Physics I }\\[1.5cm] % Major heading such as course name


    \begin{figure}
      \includegraphics[width=\linewidth]{asu.png}
    \end{figure}


    \HRule \\[0.4cm]
    { \huge \bfseries Homework 6}\\[0.4cm] 
    \HRule \\[1.5cm]
    
    \textbf{Behnam Amiri}

    \bigbreak

    \textbf{Prof: Cecilia Lunardini}

    \bigbreak


    \textbf{{\large \today}\\[2cm]}

    \vfill % Fill the rest of the page with whitespace

  \end{titlepage}

  \begin{enumerate}
    \item  Calculate the determinant of the matrix. $$A = 
      \begin{pmatrix}
      1 & 3 & 2 \\
      -1 & 2 & 1 \\
      2 & 4 & 2
      \end{pmatrix} $$

      $det(A)=2(-1)^{3+1}
      \begin{vmatrix}
        3 & 2 \\
        2 & 1
      \end{vmatrix}
      +
      4(-1)^{3+2}
      \begin{vmatrix}
        1 & 2 \\
        -1 & 1
      \end{vmatrix}
      +
      2(-1)^{3+3}
      \begin{vmatrix}
        1 & 3 \\
        -1 & 2
      \end{vmatrix}
      $

      $
      = 2(3-4)-4(1+2)+2(2+3)
      = 2(-1)-4(3)+2(5)
      $

      $
      \Longrightarrow
      det(A)=-4
      $


    \item  Calculate the determinants of the following matrices. $$M = 
      \begin{pmatrix}
      a & b & 0 & 0 \\
      0 & a & b & 0 \\
      0 & 0 & a & b \\
      c & 0 & 0 & a
      \end{pmatrix} \hskip 2truecm
      N=\begin{pmatrix}
        1 & 1 & 1 & 1 \\
        a & 1 & 1 & 1 \\
        b &a & 1 & 1 \\
        c & b & a & 1
      \end{pmatrix}
      $$

      $
      det(M)=
      a(-1)^{1+1}
      \begin{vmatrix}
        a & b & 0 \\
        0 & a & b \\
        0 & 0 & a 
      \end{vmatrix}
      +
      c(-1)^{4+1}
      \begin{vmatrix}
        b & 0 & 0 \\
        a & b & 0 \\
        0 & a & b
      \end{vmatrix}
      $

      $=a\left[a(-1)^{1+1}
      \begin{vmatrix}
        a & b \\
        0 & a
      \end{vmatrix}\right]
      -c\left[b(-1)^{1+1}
      \begin{vmatrix}
        b & 0 \\
        a & b
      \end{vmatrix}\right]
      =a\left[a(a^2)\right]-c\left[b(b^2)\right]
      $

      $
      \Longrightarrow
      det(M)=a^4-b^3c
      $

      \bigbreak

      $
      det(N)=b(-1)^{3+1}
      \begin{vmatrix}
        1 & 1 & 1 \\
        1 & 1 & 1 \\
        b & a & 1
      \end{vmatrix}
      +a(-1)^{3+2}
      \begin{vmatrix}
        1 & 1 & 1 \\
        a & 1 & 1 \\
        c & a & 1
      \end{vmatrix}
      +(-1)^{3+3}
      \begin{vmatrix}
        1 & 1 & 1 \\
        a & 1 & 1 \\
        c & b & 1
      \end{vmatrix}
      +(-1)^{3+4}
      \begin{vmatrix}
        1 & 1 & 1 \\
        a & 1 & 1 \\
        c & b & a
      \end{vmatrix}
      $

      $=b\left[(-1)^{1+1}
      \begin{vmatrix}
        1 & 1 \\
        a & 1
      \end{vmatrix}
      +(-1)^{1+2}
      \begin{vmatrix}
        1 & 1 \\
        b & 1
      \end{vmatrix}
      +(-1)^{1+3}
      \begin{vmatrix}
        1 & 1 \\
        b & a 
      \end{vmatrix}
      \right]
      $

      $-a\left[(-1)^{1+1}
      \begin{vmatrix}
        1 & 1 \\
        a & 1
      \end{vmatrix}
      +(-1)^{1+2}
      \begin{vmatrix}
        a & 1 \\
        c & 1
      \end{vmatrix}
      +(-1)^{1+3}
      \begin{vmatrix}
        a & 1 \\
        c & a
      \end{vmatrix}
      \right]
      $

      $
      +\left[(-1)^{1+1}
      \begin{vmatrix}
        1 & 1 \\
        b & 1
      \end{vmatrix}
      +(-1)^{1+2}
      \begin{vmatrix}
        a & 1 \\
        c & 1
      \end{vmatrix}
      +(-1)^{1+3}
      \begin{vmatrix}
        a & 1 \\
        c & b
      \end{vmatrix}
      \right]
      $

      $-\left[(-1)^{1+1}
      \begin{vmatrix}
        1 & 1 \\
        b & a
      \end{vmatrix}
      +(-1)^{1+2}
      \begin{vmatrix}
        a & 1 \\
        c & a 
      \end{vmatrix}
      +(-1)^{1+3}
      \begin{vmatrix}
        a & 1 \\
        c & b
      \end{vmatrix}
      \right]
      $

      \bigbreak

      $=b\left[(1-a)-(1-b)+(a-b)\right]-a\left[(1-a)-(a-c)+(a^2-c)\right]$

      $+\left[(1-b)-(a-c)+(ab-c)\right]-\left[(a-b)-(a^2-c)+(ab-c)\right]$

      \bigbreak

      $=-a^3+2a^2-a+ab-b-a+1+a^2-ab-a+b$

      $
      \Longrightarrow
      det(N)=-a^3+3a^2-3a+1
      $


    \item Prove some properties of determinants, and specifically properties (iii), (iv) and (v) listed in the textbook on page 262 (sec. 8.9). \\
    \emph{(Note: the textbook refers to a formula in chapter 16 for the proofs, however this is not necessary. You can do the proofs using a general formula for the Laplace expansion, e.g. eq. (8.47) of the book. Keep your proof as general as possible. Ideally, you should do the proof for a matrix of any generic dimension. However, proofs for a 3 by 3 generic matrix will be accepted.) }


    \item Establish which of the matrices below are Hermitian. $$A = 
      \begin{pmatrix}
        1 & i & 3 & 6 \\
        -i & 2-i & 1+i  & 0 \\
        3 & 1-i & -1 & 1 \\
        6 & 0 & 1 & 0 \\
      \end{pmatrix}\hskip 0.2truecm 
      B=\begin{pmatrix}
        1 & i & 3 & 6 \\
        -i & 2 & 1+i  & 0 \\
        3 & 1-i & -1 & 1-\sqrt{2}i \\
        6 & 0 & 1-\sqrt{2}i & 0 \\
      \end{pmatrix}
      \hskip 0.2truecm 
      C=\begin{pmatrix}
        1 & i & 3 & 6 \\
        -i & 2 & 1+i  & 0 \\
        3 & 1-i & -1 & 1-\sqrt{2}i \\
        6 & 0 & 1+\sqrt{2}i & 4 \\
      \end{pmatrix}
      $$

    \item  Given two matrices $A$ and $B$,  prove that (assuming $A^{-1}$ and $B^{-1}$ exist),
    $(AB)^{-1} = B^{-1}A^{-1} $.  


    \item  Consider the matrices below. Establish which of them can be inverted (i.e., the inverse exists) and calculate the inverses: 

      $$A_1 = 
      \begin{pmatrix}
        1 & 2 \\
        2 & -1
      \end{pmatrix}\hskip 0.2truecm 
      A_2=\begin{pmatrix}
        3 & 0 \\
        0 & 1
      \end{pmatrix}\hskip 0.2truecm
      A_3=\begin{pmatrix}
        1 & 1 \\
        2 & 3
      \end{pmatrix}
      $$

      $$A_4 = 
      \begin{pmatrix}
        1 & -4 & 2 \\
        0 & 2 & -1 \\
        0 & 0 & 5
      \end{pmatrix}\hskip 0.2truecm 
      A_5=\begin{pmatrix}
        -2 & 0 & 0 \\
        0 & 1 & 0 \\
        0 & 0 & 3
      \end{pmatrix}\hskip 0.2truecm
      A_6=\begin{pmatrix}
        1 & -1 & 3 \\
        1 & 1 & 2 \\
        2 & 0 & 7
      \end{pmatrix}
      $$
    
    
    \item  Consider the matrix written below, where $k$ is a real number (i.e., a parameter). (i) Find the rank of the matrix for varying values of $k$. (ii) establish for what values of $k$ the inverse exists. 
      $$
      Q = 
      \begin{pmatrix}
      2 & 0 & 5 \\
      -4k & 4k-1 & k-2 \\
      0 & 1-4k & 0
      \end{pmatrix}
      $$

  \end{enumerate}

\end{document}
