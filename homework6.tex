\documentclass[fleqn]{article}
\oddsidemargin 0.0in
\textwidth 6.0in
\thispagestyle{empty}
\usepackage{import}
\usepackage{amsmath}
\usepackage{graphicx}
\usepackage[english]{babel}
\usepackage[utf8x]{inputenc}
\usepackage{float}
\usepackage[colorinlistoftodos]{todonotes}


\definecolor{hwColor}{HTML}{AD53BA}

\begin{document}

  \begin{titlepage}

    \newcommand{\HRule}{\rule{\linewidth}{0.5mm}} 

    \center
    
    \textsc{\LARGE Arizona State University}\\[1.5cm] % Name of your university/college

    \textsc{\LARGE Mathematical Methods For Physics I }\\[1.5cm] % Major heading such as course name


    \begin{figure}
      \includegraphics[width=\linewidth]{asu.png}
    \end{figure}


    \HRule \\[0.4cm]
    { \huge \bfseries Homework 6}\\[0.4cm] 
    \HRule \\[1.5cm]
    
    \textbf{Behnam Amiri}

    \bigbreak

    \textbf{Prof: Cecilia Lunardini}

    \bigbreak


    \textbf{{\large \today}\\[2cm]}

    \vfill % Fill the rest of the page with whitespace

  \end{titlepage}

  \begin{enumerate}
    \item  Calculate the determinant of the matrix. $$A = 
      \begin{pmatrix}
      1 & 3 & 2 \\
      -1 & 2 & 1 \\
      2 & 4 & 2
      \end{pmatrix} $$

      \textcolor{hwColor}{
        $det(A)=2(-1)^{3+1}
        \begin{vmatrix}
          3 & 2 \\
          2 & 1
        \end{vmatrix}
        +
        4(-1)^{3+2}
        \begin{vmatrix}
          1 & 2 \\
          -1 & 1
        \end{vmatrix}
        +
        2(-1)^{3+3}
        \begin{vmatrix}
          1 & 3 \\
          -1 & 2
        \end{vmatrix}
        $
      }

      \textcolor{hwColor}{
        $det(A)=2(-1)^{3+1}
        \begin{vmatrix}
          3 & 2 \\
          2 & 1
        \end{vmatrix}
        +
        4(-1)^{3+2}
        \begin{vmatrix}
          1 & 2 \\
          -1 & 1
        \end{vmatrix}
        +
        2(-1)^{3+3}
        \begin{vmatrix}
          1 & 3 \\
          -1 & 2
        \end{vmatrix}
        $
      }

      \textcolor{hwColor}{
        $
        = 2(3-4)-4(1+2)+2(2+3)
        = 2(-1)-4(3)+2(5)
        $  
      }

      \textcolor{hwColor}{
        $
        \Longrightarrow
        det(A)=-4
        $
      }

    \item  Calculate the determinants of the following matrices. $$M = 
      \begin{pmatrix}
      a & b & 0 & 0 \\
      0 & a & b & 0 \\
      0 & 0 & a & b \\
      c & 0 & 0 & a
      \end{pmatrix} \hskip 2truecm
      N=\begin{pmatrix}
        1 & 1 & 1 & 1 \\
        a & 1 & 1 & 1 \\
        b &a & 1 & 1 \\
        c & b & a & 1
      \end{pmatrix}
      $$

      \textcolor{hwColor}{
        $
        det(M)=
        a(-1)^{1+1}
        \begin{vmatrix}
          a & b & 0 \\
          0 & a & b \\
          0 & 0 & a 
        \end{vmatrix}
        +
        c(-1)^{4+1}
        \begin{vmatrix}
          b & 0 & 0 \\
          a & b & 0 \\
          0 & a & b
        \end{vmatrix}
        $
      }

      \textcolor{hwColor}{
        $=a\left[a(-1)^{1+1}
        \begin{vmatrix}
          a & b \\
          0 & a
        \end{vmatrix}\right]
        -c\left[b(-1)^{1+1}
        \begin{vmatrix}
          b & 0 \\
          a & b
        \end{vmatrix}\right]
        =a\left[a(a^2)\right]-c\left[b(b^2)\right]
        $
      }

      \textcolor{hwColor}{
        $
        \Longrightarrow
        det(M)=a^4-b^3c
        $
      }

      \bigbreak

      \textcolor{hwColor}{
        $
        det(N)=b(-1)^{3+1}
        \begin{vmatrix}
          1 & 1 & 1 \\
          1 & 1 & 1 \\
          b & a & 1
        \end{vmatrix}
        +a(-1)^{3+2}
        \begin{vmatrix}
          1 & 1 & 1 \\
          a & 1 & 1 \\
          c & a & 1
        \end{vmatrix}
        +(-1)^{3+3}
        \begin{vmatrix}
          1 & 1 & 1 \\
          a & 1 & 1 \\
          c & b & 1
        \end{vmatrix}
        +(-1)^{3+4}
        \begin{vmatrix}
          1 & 1 & 1 \\
          a & 1 & 1 \\
          c & b & a
        \end{vmatrix}
        $
      }

      \textcolor{hwColor}{
        $=b\left[(-1)^{1+1}
        \begin{vmatrix}
          1 & 1 \\
          a & 1
        \end{vmatrix}
        +(-1)^{1+2}
        \begin{vmatrix}
          1 & 1 \\
          b & 1
        \end{vmatrix}
        +(-1)^{1+3}
        \begin{vmatrix}
          1 & 1 \\
          b & a 
        \end{vmatrix}
        \right]
        $
      }

      \textcolor{hwColor}{
        $-a\left[(-1)^{1+1}
        \begin{vmatrix}
          1 & 1 \\
          a & 1
        \end{vmatrix}
        +(-1)^{1+2}
        \begin{vmatrix}
          a & 1 \\
          c & 1
        \end{vmatrix}
        +(-1)^{1+3}
        \begin{vmatrix}
          a & 1 \\
          c & a
        \end{vmatrix}
        \right]
        $
      }

      \textcolor{hwColor}{
        $
        +\left[(-1)^{1+1}
        \begin{vmatrix}
          1 & 1 \\
          b & 1
        \end{vmatrix}
        +(-1)^{1+2}
        \begin{vmatrix}
          a & 1 \\
          c & 1
        \end{vmatrix}
        +(-1)^{1+3}
        \begin{vmatrix}
          a & 1 \\
          c & b
        \end{vmatrix}
        \right]
        $
      }

      \textcolor{hwColor}{
        $-\left[(-1)^{1+1}
        \begin{vmatrix}
          1 & 1 \\
          b & a
        \end{vmatrix}
        +(-1)^{1+2}
        \begin{vmatrix}
          a & 1 \\
          c & a 
        \end{vmatrix}
        +(-1)^{1+3}
        \begin{vmatrix}
          a & 1 \\
          c & b
        \end{vmatrix}
        \right]
        $
      }

      \bigbreak

      \textcolor{hwColor}{$=b\left[(1-a)-(1-b)+(a-b)\right]-a\left[(1-a)-(a-c)+(a^2-c)\right]$}

      \textcolor{hwColor}{$=b\left[(1-a)-(1-b)+(a-b)\right]-a\left[(1-a)-(a-c)+(a^2-c)\right]$}

      \textcolor{hwColor}{$=b\left[(1-a)-(1-b)+(a-b)\right]-a\left[(1-a)-(a-c)+(a^2-c)\right]$}

      \textcolor{hwColor}{$+\left[(1-b)-(a-c)+(ab-c)\right]-\left[(a-b)-(a^2-c)+(ab-c)\right]$}

      \textcolor{hwColor}{$=-a^3+2a^2-a+ab-b-a+1+a^2-ab-a+b$}
   
      \textcolor{hwColor}{    
        $
        \Longrightarrow
        det(N)=-a^3+3a^2-3a+1
        $
      }


    \item Prove some properties of determinants, and specifically properties (iii), (iv) and (v) listed in the textbook on page 262 (sec. 8.9). \\
    \emph{(Note: the textbook refers to a formula in chapter 16 for the proofs, however this is not necessary. You can do the proofs using a general formula for the Laplace expansion, e.g. eq. (8.47) of the book. Keep your proof as general as possible. Ideally, you should do the proof for a matrix of any generic dimension. However, proofs for a 3 by 3 generic matrix will be accepted.) }

    $(i) \begin{vmatrix}
      A^T
    \end{vmatrix}=
    \begin{vmatrix}
      A
    \end{vmatrix}
    $

    \textcolor{hwColor}{
      Assume
      $
      A=\begin{pmatrix}
        a & b \\
        c & d \\
      \end{pmatrix}
      $
      , then
      $
      A^T\begin{pmatrix}
        a & c \\
        b & d \\
      \end{pmatrix}
      $
    }

    \textcolor{hwColor}{
      $
        |A|=(ad-cb)
      $
      , and
      $
      |A^T|=(ad-cb)
      $
    }

    \textcolor{hwColor}{
      therefore, 
      $
      \begin{vmatrix}
        A^T
      \end{vmatrix}=
      \begin{vmatrix}
        A
      \end{vmatrix}
      $
    }

    \bigbreak

    $
    (ii) \begin{vmatrix}
      A^{\dagger}
    \end{vmatrix}=\begin{vmatrix}
      (A^*)^T
    \end{vmatrix}=\begin{vmatrix}
      A^*
    \end{vmatrix}=\begin{vmatrix}
      A
    \end{vmatrix}^*
    $

    \bigbreak

    $(iii)$ Interchanging two rows or two columns. If two rows (columns) of $A$ are interchanged, its determinant changes sign but is unaltered in magnitude.

    \textcolor{hwColor}{
      Assume
      $
      A_1=\begin{pmatrix}
        a & b \\
        c & d \\
      \end{pmatrix}
      $
      , and
      $
      A_2=\begin{pmatrix}
        c & d \\
        a & b \\
      \end{pmatrix}
      $
    }

    \textcolor{hwColor}{
      $
      det(A_1)=ad-cb
      $
      , and
      $
      det(A_2)=cb-ad
      $
    }

    \textcolor{hwColor}{
      By swapping rows, their determinants change sign.  $det(A_1)=-det(A_2)$
    }

    \bigbreak

    $(iv)$ Removing factors. If all the elements of a single row (column) of $A$ have a common factor, $\lambda$, then this factor may be removed;  the value of the determinant is given by the product of the remaining determinant and $\lambda$.Clearly this implies that if all the elements of any row (column) are zero then $\begin{vmatrix}
      A
    \end{vmatrix}=0$. It also follows that if every element of the $N \times N$ matrix $A$ is multiplied by a constant factor $\lambda$ then

    $\begin{vmatrix}
      \lambda A
    \end{vmatrix}= \lambda^N \begin{vmatrix}
      A
    \end{vmatrix}
    $

    \textcolor{hwColor}{
      $
      A=\begin{pmatrix}
        \lambda a & \lambda b \\
        c & d \\
      \end{pmatrix}
      $
      then 
      $
      det(A)= (\lambda ad)-(\lambda bc)=\lambda (ad-bc)=\lambda \begin{vmatrix}
        A
      \end{vmatrix}
      $
    }

    \bigbreak

    $(v)$ Identical rows or columns. If any two rows (columns) of $A$ are identical or are multiples of one another, then it can be shown that $\begin{vmatrix}
      A
    \end{vmatrix}=0$.

    \textcolor{hwColor}{
      $
      A=\begin{pmatrix}
        a & b & c \\
        a & b & c \\
        d & e & f \\
      \end{pmatrix}
      $
    }

    \textcolor{hwColor}{
      $
      det(A)=a(-1)^{1+1}\begin{pmatrix}
        b & c \\
        e & f \\
      \end{pmatrix}
      +b(-1)^{1+2}\begin{pmatrix}
        a & c \\
        d & f \\
      \end{pmatrix}
      +c(-1)^{1+3}\begin{pmatrix}
        a & b \\
        d & e \\
      \end{pmatrix}
      $
    }

    \textcolor{hwColor}{
      $
      det(A)=a(bf-ec)-b(af-cd)+c(ae-bd)=0
      $
    }

    \bigbreak

    \item Establish which of the matrices below are Hermitian. $$A = 
      \begin{pmatrix}
        1 & i & 3 & 6 \\
        -i & 2-i & 1+i  & 0 \\
        3 & 1-i & -1 & 1 \\
        6 & 0 & 1 & 0 \\
      \end{pmatrix}\hskip 0.2truecm 
      B=\begin{pmatrix}
        1 & i & 3 & 6 \\
        -i & 2 & 1+i  & 0 \\
        3 & 1-i & -1 & 1-\sqrt{2}i \\
        6 & 0 & 1-\sqrt{2}i & 0 \\
      \end{pmatrix}
      \hskip 0.2truecm 
      C=\begin{pmatrix}
        1 & i & 3 & 6 \\
        -i & 2 & 1+i  & 0 \\
        3 & 1-i & -1 & 1-\sqrt{2}i \\
        6 & 0 & 1+\sqrt{2}i & 4 \\
      \end{pmatrix}
      $$

      \textcolor{hwColor}{Hermitian matrices are square matrices whose conjugate transpose is the same as the original matrix.}

      \textcolor{hwColor}{
        $
        A^*=\begin{pmatrix}
          1 & -i & 3 & 6 \\
          i & 2+i & 1-i & 0 \\
          3 & 1+i & -1 & 1 \\
          6 & 0 & 1 & 0
        \end{pmatrix}
        \hskip 2truecm
        (A^*)^T=\begin{pmatrix}
          1 & i & 3 & 6 \\
          -i & 2+i & 1+i  & 0 \\
          3 & 1-i & -1 & 1 \\
          6 & 0 & 1 & 0 \\
        \end{pmatrix}
        $ 
      }

      \textcolor{hwColor}{$A \ne (A^*)^T\Longrightarrow$ A is NOT a hermitian matrix.}

      \bigbreak

      \textcolor{hwColor}{
        $
        B^*=\begin{pmatrix}
          1 & -i & 3 & 6 \\
          i & 2 & 1-i  & 0 \\
          3 & 1+i & -1 & 1+\sqrt{2}i \\
          6 & 0 & 1+\sqrt{2}i & 0 \\
        \end{pmatrix}
        \hskip 2truecm
        (B^*)^T=\begin{pmatrix}
          1 & i & 3 & 6 \\
          -i & 2 & 1+i & 0 \\
          3 & 1-i & -1 & 1+\sqrt{2}i \\
          6 & 0 & 1+\sqrt{2}i & 0
        \end{pmatrix}
        $
      }

      \textcolor{hwColor}{$B=(B^*)^T\Longrightarrow$ B is a hermitian matrix.}

      \bigbreak

      \textcolor{hwColor}{
        $
        C^*=\begin{pmatrix}
          1 & -i & 3 & 6 \\
          i & 2 & 1-i & 0 \\
          3 & 1+i & -1 & 1+\sqrt{2}i \\
          6 & 0 & 1-\sqrt{2}i & 4 \\
        \end{pmatrix}
        \hskip 2truecm
        (C^*)^T=\begin{pmatrix}
          1 & i & 3 & 6 \\
          -i & 2 & 1+i & 0 \\
          3 & 1-i & -1 & 1-\sqrt{2}i \\
          6 & 0 & 1+\sqrt{2}i & 4
        \end{pmatrix}
        $
      }

      \textcolor{hwColor}{$C=(C^*)^T\Longrightarrow$ C is a hermitian matrix.}


    \item  Given two matrices $A$ and $B$,  prove that (assuming $A^{-1}$ and $B^{-1}$ exist),
    $(AB)^{-1} = B^{-1}A^{-1} $.  

    \textcolor{hwColor}{$(AB)(AB)^{-1}=I$}

    \textcolor{hwColor}{$A^{-1}(AB)(AB)^{-1}=A^{-1}I$}

    \textcolor{hwColor}{$I(B)(AB)^{-1}=A^{-1}$}

    \textcolor{hwColor}{$B(AB)^{-1}=A^{-1}$}

    \textcolor{hwColor}{$B^{-1}B(AB)^{-1}=B^{-1}A^{-1}$}

    \textcolor{hwColor}{$I(AB)^{-1}=B^{-1}A^{-1}$}

    \textcolor{hwColor}{$(AB)^{-1}=B^{-1}A^{-1}$}

    \bigbreak

    \item  Consider the matrices below. Establish which of them can be inverted (i.e., the inverse exists) and calculate the inverses: 

      $$A_1 = 
      \begin{pmatrix}
        1 & 2 \\
        2 & -1
      \end{pmatrix}\hskip 0.2truecm 
      A_2=\begin{pmatrix}
        3 & 0 \\
        0 & 1
      \end{pmatrix}\hskip 0.2truecm
      A_3=\begin{pmatrix}
        1 & 1 \\
        2 & 3
      \end{pmatrix}
      $$

      $$A_4 = 
      \begin{pmatrix}
        1 & -4 & 2 \\
        0 & 2 & -1 \\
        0 & 0 & 5
      \end{pmatrix}\hskip 0.2truecm 
      A_5=\begin{pmatrix}
        -2 & 0 & 0 \\
        0 & 1 & 0 \\
        0 & 0 & 3
      \end{pmatrix}\hskip 0.2truecm
      A_6=\begin{pmatrix}
        1 & -1 & 3 \\
        1 & 1 & 2 \\
        2 & 0 & 7
      \end{pmatrix}
      $$

      \textcolor{hwColor}{
        $
        det(A_1)=(-1-(4))=-5
        \hskip 1truecm
        A_1^{-1}=\dfrac{adj(A_1)}{det(A_1)}=\dfrac{
          \begin{vmatrix}
            -1 & -2 \\
            -2 & 1
          \end{vmatrix}
        }{-5}=\begin{pmatrix}
          \dfrac{1}{5} & \dfrac{2}{5} \\
          \\
          \dfrac{2}{5} & -\dfrac{1}{5} \\
        \end{pmatrix}
        $
      }

      \textcolor{hwColor}{
        $
        det(A_2)=(3-0)=3
        \hskip 1truecm
        A_2^{-1}=\dfrac{adj(A_2)}{det(A_2)}=\dfrac{
          \begin{vmatrix}
            1 & 0 \\
            0 & 3
          \end{vmatrix}
        }{3}=\begin{pmatrix}
          \dfrac{1}{3} & 0 \\
          0 & 1 \\
        \end{pmatrix}
        $
      }

      \textcolor{hwColor}{
        $
        det(A_3)=(3-2)=1
        \hskip 1truecm
        A_3^{-1}=\dfrac{adj(A_3)}{det(A_3)}=\dfrac{
          \begin{vmatrix}
            3 & -1 \\
            -2 & 1 \\
          \end{vmatrix}
        }{1}=
        \begin{pmatrix}
          3 & -1 \\
          -2 & 1 \\
        \end{pmatrix}
        $
      }

      \textcolor{hwColor}{
        $
        det(A_4)=(2-5)=10
        \hskip 1truecm
        A_4^{-1}=\dfrac{adj(A_4)}{det(A_4)}=\dfrac{
          \begin{vmatrix}
           10 & 20 & 0 \\
           0 & 5 & 1 \\
           0 & 0 & 2 \\
          \end{vmatrix}
        }{10}=\begin{pmatrix}
          1 & 2 & 0 \\
          \\
          0 & \dfrac{1}{2}& \dfrac{1}{10} \\
          \\
          0 & 0 & \dfrac{1}{5}\\
        \end{pmatrix}
        $
      }

      \textcolor{hwColor}{
        $
        det(A_5)=(-2)(3-0)=-6
        \hskip 1truecm
        A_5^{-1}=\dfrac{adj(A_5)}{det(A_5)}=\dfrac{
          \begin{vmatrix}
           3 & 0 & 0 \\
           0 & -6 & 0 \\
           0 & 0 & -2 \\
          \end{vmatrix}
        }{-6}=\begin{pmatrix}
          -\dfrac{1}{2}& 0 & 0 \\
          0 & 1 & 0 \\
          0 & 0 & \dfrac{1}{3}\\
        \end{pmatrix}
        $
      }

      \textcolor{hwColor}{
        $
        det(A_6)=2(-2-3)+7(1-(-1))=4
        \hskip 1truecm
        A_6^{-1}=\dfrac{adj(A_6)}{det(A_6)}=\dfrac{
          \begin{vmatrix}
           7 & 7 & -5 \\
           -3 & 1 & 1 \\
           -2 & -2 & 2 \\
          \end{vmatrix}
        }{4}=\begin{pmatrix}
         \dfrac{7}{4} & \dfrac{7}{4} & -\dfrac{5}{4} \\
         \\
         -\dfrac{3}{4} & \dfrac{1}{4} & \dfrac{1}{4} \\
         \\
         -\dfrac{1}{2} & -\dfrac{1}{2} & \dfrac{1}{2} \\
        \end{pmatrix}
        $
      }

    \bigbreak
    
    \item  Consider the matrix written below, where $k$ is a real number (i.e., a parameter). (i) Find the rank of the matrix for varying values of $k$. (ii) establish for what values of $k$ the inverse exists. 
      $$
      Q = 
      \begin{pmatrix}
      2 & 0 & 5 \\
      -4k & 4k-1 & k-2 \\
      0 & 1-4k & 0
      \end{pmatrix}
      $$

      \textcolor{hwColor}{
        (i) When a matrix is reduced to the identity matrix, then the size of the reduced matrix is the rank of that matrix.
      }

      \textcolor{hwColor}{
        $
        Q = 
          \begin{pmatrix}
          2 & 0 & 5 \\
          -4k & 4k-1 & k-2 \\
          0 & 1-4k & 0
          \end{pmatrix}
          \hspace{2cm} \dfrac{1}{2}R_1 \longrightarrow R_1 \hspace{2cm}
        $
      }

      \textcolor{hwColor}{
        $
          Q=\begin{pmatrix}
            1 & 0 & \dfrac{5}{2} \\
            -4k & 4k-1 & k-2 \\
            0 & 1-4k & 0 \\
          \end{pmatrix}
          \hspace{2cm} R_2+4kR_1 \longrightarrow R_2 \hspace{1cm}
        $
      }

      \textcolor{hwColor}{
        $
          Q=\begin{pmatrix}
          1 & 0 & \dfrac{5}{2} \\
          0 & 4k-1 & 11k-2 \\
          0 & 1-4k & 0 \\
        \end{pmatrix}
        \hspace{2cm} -\dfrac{5}{2}C_1+C_3 \longrightarrow C_3
        $
      }

      \textcolor{hwColor}{
        $
          Q=\begin{pmatrix}
          1 & 0 & 0 \\
          0 & 4k-1 & 11k-2 \\
          0 & 1-4k & 0 \\
        \end{pmatrix}
        \hspace{2cm} R_2+R_3 \longrightarrow R_3
        $
      }

      \textcolor{hwColor}{
        $
          Q=\begin{pmatrix}
          1 & 0 & 0 \\
          0 & 4k-1 & 11k-2 \\
          0 & 0 & 11k-2 \\
        \end{pmatrix}
        \hspace{2cm} R_2+(-R_3) \longrightarrow R_2
        $
      }

      \textcolor{hwColor}{
        $
          Q=\begin{pmatrix}
          1 & 0 & 0 \\
          0 & 4k-1 & 0 \\
          0 & 0 & 11k-2 \\
        \end{pmatrix}
        $
      }

      \textcolor{hwColor}{
        Assume
        $
          Q=\begin{pmatrix}
          1 & 0 & 0 \\
          0 & 4k-1 & 0 \\
          0 & 0 & 11k-2 \\
        \end{pmatrix}=\begin{pmatrix}
          1 & 0 & 0 \\
          0 & 1 & 0 \\
          0 & 0 & 1 \\
        \end{pmatrix}
        $
        , then 
        $
         \longrightarrow
         \begin{cases}
           4k-1=1 \\
           11k-2=1  
         \end{cases}
         \longrightarrow
         \begin{cases}
           k=\dfrac{1}{2} \\
           \\
           k= \dfrac{3}{11} 
         \end{cases}
        $
      }

      \bigbreak

      \textcolor{hwColor}{
        Case 1: 
        $
           k=\dfrac{1}{2}
           \\
           Q=\begin{pmatrix}
             1 & 0 & 0 \\
             0 & 1 & 0 \\
             0 & 0 & \dfrac{7}{2}
           \end{pmatrix}
           \hspace{1cm} \dfrac{2}{7}R_3 \longrightarrow R_3
           \hspace{1cm} Q=\begin{pmatrix}
            1 & 0 & 0 \\
            0 & 1 & 0 \\
            0 & 0 & 1
           \end{pmatrix}
        $
      }

      \bigbreak

      \textcolor{hwColor}{
        Case 2: 
        $
           k=\dfrac{3}{11}
           \\
           Q=\begin{pmatrix}
             1 & 0 & 0 \\
             0 & \dfrac{1}{11} & 0 \\
             0 & 0 & 1
           \end{pmatrix}
           \hspace{1cm} 11R_2 \longrightarrow R_2
           \hspace{1cm} Q=\begin{pmatrix}
            1 & 0 & 0 \\
            0 & 1 & 0 \\
            0 & 0 & 1
           \end{pmatrix}
        $
      }

      \textcolor{hwColor}{
          Hence, when 
        $
         \longrightarrow
         \begin{cases}
          k=\dfrac{1}{2} \\
          \\
          k= \dfrac{3}{11} 
         \end{cases}
        $
        the rank of matrix $Q$ is 3.
      }
      
      \textcolor{hwColor}{
        \rule{16cm}{0.4pt}
      }

      \bigbreak

      \textcolor{hwColor}{
        Assume
        $
          Q=\begin{pmatrix}
          1 & 0 & 0 \\
          0 & 4k-1 & 0 \\
          0 & 0 & 11k-2 \\
        \end{pmatrix}=\begin{pmatrix}
          1 & 0 & 0 \\
          0 & 1 & 0 \\
          0 & 0 & 0 \\
        \end{pmatrix}
        $
        , then 
        $
         \longrightarrow
         \begin{cases}
           4k-1=1 \\
           11k-2=0  
         \end{cases}
         \longrightarrow
         \begin{cases}
           k=\dfrac{1}{2} \\
           \\
           k= \dfrac{2}{11} 
         \end{cases}
        $
      }

      \textcolor{hwColor}{
        Case 1: 
        $
           k=\dfrac{1}{2}
           \\
           Q=\begin{pmatrix}
             1 & 0 & 0 \\
             0 & 1 & 0 \\
             0 & 0 & \dfrac{7}{2}
           \end{pmatrix}
           \hspace{0.5cm} \dfrac{2}{7}R_3 \longrightarrow R_3
           \hspace{0.5cm} Q=\begin{pmatrix}
            1 & 0 & 0 \\
            0 & 1 & 0 \\
            0 & 0 & 1
           \end{pmatrix}
        $
        the rank of the matrix is 3 when $k=\dfrac{1}{2}$.
      }

      \bigbreak

      \textcolor{hwColor}{
        Case 2: 
        $
           k=\dfrac{2}{11}
           \\
           Q=\begin{pmatrix}
             1 & 0 & 0 \\
             0 & -\dfrac{3}{11} & 0 \\
             0 & 0 & 0
           \end{pmatrix}
           \hspace{0.1cm} -\dfrac{11}{3}R_2 \longrightarrow R_2
           \hspace{0.5cm} Q=\begin{pmatrix}
            1 & 0 & 0 \\
            0 & 1 & 0 \\
            0 & 0 & 0
           \end{pmatrix}
        $
        the rank of the matrix is 2 when $k=\dfrac{2}{11}$.
      }

      \textcolor{hwColor}{
        \rule{16cm}{0.4pt}
      }

      \bigbreak

      \textcolor{hwColor}{
        Assume
        $
          Q=\begin{pmatrix}
          1 & 0 & 0 \\
          0 & 4k-1 & 0 \\
          0 & 0 & 11k-2 \\
        \end{pmatrix}=\begin{pmatrix}
          1 & 0 & 0 \\
          0 & 0 & 0 \\
          0 & 0 & 0 \\
        \end{pmatrix}
        $
        , then 
        $
         \longrightarrow
         \begin{cases}
           4k-1=0 \\
           11k-2=0  
         \end{cases}
         \longrightarrow
         \begin{cases}
           k=\dfrac{1}{4} \\
           \\
           k= \dfrac{2}{11} 
         \end{cases}
        $
        Since there is not a $k$ value such that matrix $Q$ can be written as an $1 \times 1$ identity matrix, therefore, matrix Q can only have rank 2 and 3 for variying value of $k$. 
      }

      \bigbreak

      \textcolor{hwColor}{
        \rule{16cm}{0.4pt}
      }

      \bigbreak

      \textcolor{hwColor}{
        (ii) The inverse of matrix $Q$ exists iff the determinant of $Q\neq0$.
      }

      \textcolor{hwColor}{
        $
        det(Q)=(1-4k)(-1)^{3+2}\begin{vmatrix}
          2 & 5 \\
          -4k & k-2 \\
        \end{vmatrix}
        =
        88k^2-38k+4
        $
      }

      \textcolor{hwColor}{
        $
        88k^2-38k+4=0
        \longrightarrow
        \begin{cases}
          k=\dfrac{1}{4} \\
          \\
          k= \dfrac{2}{11} 
        \end{cases}
        $
      }

      \textcolor{hwColor}{
        For $k=\dfrac{1}{4}, det(Q)=0 \Longrightarrow Q^{-1}$ does not exist.  
      }

      \textcolor{hwColor}{
        For $k=\dfrac{2}{11}, det(Q)=0 \Longrightarrow Q^{-1}$ does not exist.  
      }

      \textcolor{hwColor}{
        Therefore, the $Q^{-1}$ exists when $k\neq\dfrac{1}{4}$ and $k\neq\dfrac{2}{11}$  .  
      }



  \end{enumerate}

\end{document}
