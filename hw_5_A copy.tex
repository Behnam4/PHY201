\documentclass[fleqn]{article}
\oddsidemargin 0.0in
\textwidth 6.0in
\thispagestyle{empty}
\usepackage{import}
\usepackage{amsmath}
\usepackage{graphicx}


\begin{document}

%useful notation: \mathbf{A}=\mathbf{\hat{x}}+2\mathbf{\hat{y}-}2\mathbf{\hat{z}, \quad \quad B}=3\mathbf{\hat{x}}+\mathbf{\hat{y}+}2\mathbf{\hat{z}, \quad \quad C}=4\mathbf{\hat{x}}-\mathbf{\hat{y}+\hat{z}.}

{\bf Homework 5 - Part A}\\
Homework assigned on a given day is always due on the following Tuesday at noon. Electronic submission via Canvas is required. Single file, .pdf or .jpeg only.  A scannerized copy of a handwritten file is acceptable. Typeset solutions are welcome as well. The student is responsible for the readability of the file. 

\begin{enumerate}


\item Consider the vectors: $\mathbf{a}=( \mathbf{\hat{x}} - \mathbf{\hat{y}} ) / \sqrt{2} $, $\mathbf{b}=( \mathbf{\hat{x}} + \mathbf{\hat{y}} + 2 \mathbf{\hat{z}} ) / \sqrt{6}$ and $\mathbf{c}=(\mathbf{\hat{x}}+\mathbf{\hat{y}}-\mathbf{\hat{z}})/\sqrt{3}$. 
% . 
\begin{enumerate}
\item Prove that they are a basis for a three dimensional cartesian space, and check if they are an \emph{orthogonal} basis. 
\\

So to show that the set of vectors forms a basis for $R^3$ we only need to show that all possible vectors in $R^3$ can be made from some linear combination of these vectors. This is equivalent to putting these three vectors into a matrix, and showing it row reduces to the identity matrix (since the basis vectors of $R^3$ are (1,00), (0,1,0), and (0,0,1)).
\\
So what we need to do first is make a, b, c the columns of a $3\times3$ matrix row reduce that matrix, so that we see it becomes the identity, and thus is a basis for $R^3$

\noindent
$
\begin{array}{lll}
  \overrightarrow{a}: \dfrac{\hat{x}-\hat{y}}{\sqrt{2}}=\dfrac{\hat{x}}{\sqrt{2}}-\dfrac{\hat{y}}{\sqrt{2}} \\
  \overrightarrow{b}: \dfrac{\hat{x}+\hat{y}+2\hat{z}}{\sqrt{6}}=\dfrac{\hat{x}}{\sqrt{6}}+\dfrac{\hat{y}}{\sqrt{6}}+\dfrac{2\hat{z}}{\sqrt{6}} \\
  \overrightarrow{c}: \dfrac{\hat{x}+\hat{y}-\hat{z}}{\sqrt{3}}=\dfrac{\hat{x}}{\sqrt{3}}+\dfrac{\hat{y}}{\sqrt{3}}-\dfrac{\hat{z}}{\sqrt{3}}
  \end{array}  
$
\bigbreak

\noindent
$\displaystyle
  \begin{pmatrix}
    \dfrac{1}{\sqrt{2}} & \dfrac{-1}{\sqrt{2}} & 0 \\
    \dfrac{1}{\sqrt{6}} & \dfrac{1}{\sqrt{6}} & \dfrac{2}{\sqrt{6}} \\ 
    \dfrac{1}{\sqrt{3}} & \dfrac{1}{\sqrt{3}} & \dfrac{-1}{\sqrt{3}} \\ 
  \end{pmatrix}
  =
  \begin{pmatrix}
    1 & -1 & 0\\
    1 & 1 & 2\\ 
    1 & 1 & -1\\ 
  \end{pmatrix}
  = 
  \begin{pmatrix}
    1 & -1 & 0\\
    0 & 2 & 2\\ 
    1 & 1 & -1\\ 
  \end{pmatrix}
  = 
  \begin{pmatrix}
    1 & -1 & 0\\
    0 & 2 & 2\\ 
    0 & 2 & -1\\ 
  \end{pmatrix}
  = 
  \begin{pmatrix}
    1 & -1 & 0\\
    0 & 1 & 1\\ 
    0 & 2 & -1\\ 
  \end{pmatrix}
$

\noindent
$\displaystyle
  =
  \begin{pmatrix}
    1 & -1 & 0\\
    0 & 1 & 1\\ 
    0 & 0 & -3\\ 
  \end{pmatrix}
  =
  \begin{pmatrix}
    1 & -1 & 0\\
    0 & 1 & 1\\ 
    0 & 0 & 1\\ 
  \end{pmatrix}
  =
  \begin{pmatrix}
    1 & -1 & 0\\
    0 & 1 & 0\\ 
    0 & 0 & 1\\ 
  \end{pmatrix}
  =
  \begin{pmatrix}
    1 & 0 & 0\\
    0 & 1 & 0\\ 
    0 & 0 & 1\\ 
  \end{pmatrix}
  =
  I_3\times_3
$

\bigbreak

To show that it is an orthogonal basis, we need to check that a, b, and c are perpendicular to each other, thus cross product of one with the other two will give 0.

\bigbreak

$
\overrightarrow{a}\times(\overrightarrow{b}\times\overrightarrow{c})=0
$

$
\overrightarrow{b}\times\overrightarrow{c}
=
\left|
\begin{matrix}
  \hat{x} & \hat{y} & \hat{z} \\
  \dfrac{1}{\sqrt{6}} & \dfrac{1}{\sqrt{6}} & \dfrac{2}{\sqrt{6}} \\ 
  \dfrac{1}{\sqrt{3}} & \dfrac{1}{\sqrt{3}} & \dfrac{-1}{\sqrt{3}} \\ 
\end{matrix}
\right|
=
\left|
\begin{matrix}
  \hat{x} & \hat{y} & \hat{z} \\
  1 & 1 & 2 \\ 
  1 & 1 & -1 \\ 
\end{matrix}
\right|
=
\hat{x}(-1-2)-\hat{y}(-1-2)+\hat{z}(1-1)
=
-3\hat{x}+3\hat{y}
$

Now $\overrightarrow{b}\times\overrightarrow{c}=\overrightarrow{V}$ vector has to be parallel with $\overrightarrow{a}$ hence, the cross product of these two vectors will give 0.

$
\overrightarrow{a}\times\overrightarrow{V}
=
\left|
\begin{matrix}
  \hat{x} & \hat{y} & \hat{z} \\
  \dfrac{1}{\sqrt{2}} & \dfrac{-1}{\sqrt{2}} & 0  \\ 
  -3 & 3 & 0 \\ 
\end{matrix}
\right|
=
\hat{x}(0-0)-\hat{y}(0-0)+\hat{z}(0-0)
=
0
$

\item Express the vectors $\mathbf{\hat{x}}, \mathbf{\hat{y}}, \mathbf{\hat{z}}$ in the basis of the vectors $\mathbf{a}, \mathbf{b}, \mathbf{c}$. 

$
\begin{array}{lll}
  \overrightarrow{a}= \dfrac{1}{\sqrt{2}}(\hat{x}-\hat{y}) \\
  \overrightarrow{b}= \dfrac{1}{\sqrt{6}}(\hat{x}+\hat{y}+2\hat{z}) \\
  \overrightarrow{c}= \dfrac{1}{\sqrt{3}}(\hat{x}-\hat{y}-\hat{z})
  \end{array}  
$

$
\begin{array}{lll}
  \hat{x}= (1,0,0) \\
  \hat{y}= (0,1,0) \\
  \hat{z}= (0,0,1)
  \end{array}  
$

From the above data we can create an augmented matrix and row-reduce.


\makeatletter
\renewcommand*\env@matrix[1][*\c@MaxMatrixCols c]{
  \hskip -\arraycolsep
  \let\@ifnextchar\new@ifnextchar
  \array{#1}}
\makeatother
\begin{equation}
  \begin{bmatrix}[ccc|c]
    \dfrac{1}{\sqrt{2}} & \dfrac{-1}{\sqrt{2}} & 0 & \overrightarrow{a} \\
    \dfrac{1}{\sqrt{6}} & \dfrac{1}{\sqrt{6}} & \dfrac{2}{\sqrt{6}} & \overrightarrow{b} \\
    \dfrac{1}{\sqrt{3}} & \dfrac{1}{\sqrt{3}} & \dfrac{-1}{\sqrt{3}} & \overrightarrow{c} \\
  \end{bmatrix}
  =
  \begin{bmatrix}[ccc|c]
    1 & 0 & 0 & \dfrac{1}{6}(3\sqrt{2}\overrightarrow{a}+\sqrt{6}\overrightarrow{b}+2\sqrt{3}\overrightarrow{c}) \\
    0 & 1 & 0 & \dfrac{1}{6}(-3\sqrt{2}\overrightarrow{a}+\sqrt{6}\overrightarrow{b}+2\sqrt{3}\overrightarrow{c}) \\
    0 & 0 & 1 & \dfrac{1}{3}(\sqrt{6}\overrightarrow{b}-\sqrt{3}\overrightarrow{c}) \\
  \end{bmatrix}
\end{equation}

$\Rightarrow$ 
$
\begin{array}{lll}
  \hat{x}= \dfrac{1}{6}(3\sqrt{2}\overrightarrow{a}+\sqrt{6}\overrightarrow{b}+2\sqrt{3}\overrightarrow{c}) \\
  \hat{y}= \dfrac{1}{6}(-3\sqrt{2}\overrightarrow{a}+\sqrt{6}\overrightarrow{b}+2\sqrt{3}\overrightarrow{c}) \\
  \hat{z}= \dfrac{1}{3}(\sqrt{6}\overrightarrow{b}-\sqrt{3}\overrightarrow{c})
\end{array}  
$


\bigbreak

\item Consider now the set of vectors $\mathbf{A}=\mathbf{\hat{x}}-\mathbf{\hat{y}}$; $\mathbf{B}=\mathbf{\hat{x}}+\mathbf{\hat{z}}$; $\mathbf{C}=\mathbf{\hat{y}}-2\mathbf{\hat{z}}$, and express them in the basis of the vectors $\mathbf{a}, \mathbf{b}, \mathbf{c}$. 

$
\begin{array}{lll}
  \hat{x}= \dfrac{1}{6}(3\sqrt{2}\overrightarrow{a}+\sqrt{6}\overrightarrow{b}+2\sqrt{3}\overrightarrow{c}) \\
  \hat{y}= \dfrac{1}{6}(-3\sqrt{2}\overrightarrow{a}+\sqrt{6}\overrightarrow{b}+2\sqrt{3}\overrightarrow{c}) \\
  \hat{z}= \dfrac{1}{3}(\sqrt{6}\overrightarrow{b}-\sqrt{3}\overrightarrow{c})
  \end{array}  
$


$\Rightarrow$ 
$
\begin{array}{lll}
  A = \hat{x}-\hat{y}=(\dfrac{1}{6}(3\sqrt{2}\overrightarrow{a}+\sqrt{6}\overrightarrow{b}+2\sqrt{3}\overrightarrow{c}))-(\dfrac{1}{6}(-3\sqrt{2}\overrightarrow{a}+\sqrt{6}\overrightarrow{b}+2\sqrt{3}\overrightarrow{c})) \\
  B = \hat{x}+\hat{z}= (\dfrac{1}{6}(3\sqrt{2}\overrightarrow{a}+\sqrt{6}\overrightarrow{b}+2\sqrt{3}\overrightarrow{c}))+(\dfrac{1}{3}(\sqrt{6}\overrightarrow{b}-\sqrt{3}\overrightarrow{c}))\\
  C = \hat{y}-2\hat{z}=(\dfrac{1}{6}(-3\sqrt{2}\overrightarrow{a}+\sqrt{6}\overrightarrow{b}+2\sqrt{3}\overrightarrow{c}))-2(\dfrac{1}{3}(\sqrt{6}\overrightarrow{b}-\sqrt{3}\overrightarrow{c}))
\end{array}  
$

\bigbreak

$\Rightarrow$ 
$
\begin{array}{lll}
  A = \sqrt{2}\overrightarrow{a} \\
  B = \dfrac{\sqrt{2}}{2}(\overrightarrow{a}+\sqrt{3}\overrightarrow{b}) \\
  C = \dfrac{-\sqrt{2}}{2}\overrightarrow{a}-\dfrac{\sqrt{6}}{2}\overrightarrow{b}+\sqrt{3}\overrightarrow{c} 
\end{array}  
$

\end{enumerate}


\item Same as the previous exercise, with $\mathbf{a}=\frac{1}{\sqrt{6}} ( \mathbf{\hat{x}} + \mathbf{\hat{y}} + 2 \mathbf{\hat{z}} ) $, $\mathbf{b}=\frac{1}{\sqrt{30}} (\mathbf{\hat{x}} - 5\mathbf{\hat{y}} + 2\mathbf{\hat{z}})$ and $\mathbf{c}= \frac{1}{\sqrt{5}} ( 2 \mathbf{\hat{x} }- \mathbf{\hat{z} })$, and 
$A=\mathbf{\hat{x}}+\mathbf{\hat{y}}+2\mathbf{\hat{z}}; B=\mathbf{\hat{x}}-\mathbf{\hat{y}}+2\mathbf{\hat{z}}; C=\mathbf{\hat{y}}-\mathbf{\hat{z}}$. 

\noindent
$
\begin{array}{lll}
  \overrightarrow{a}: \dfrac{1}{\sqrt{6}}(\hat{x}+\hat{y}+2\hat{z})=\dfrac{\hat{x}}{\sqrt{6}}+\dfrac{\hat{y}}{\sqrt{6}}+\dfrac{2\hat{z}}{\sqrt{6}} \\
  \overrightarrow{b}: \dfrac{1}{\sqrt{30}}(\hat{x}-5\hat{y}+2\hat{z})=\dfrac{\hat{x}}{\sqrt{30}}-\dfrac{5\hat{y}}{\sqrt{30}}+\dfrac{2\hat{z}}{\sqrt{30}} \\
  \overrightarrow{c}: \dfrac{1}{\sqrt{5}}(2\hat{x}-\hat{z})=\dfrac{2\hat{x}}{\sqrt{5}}-\dfrac{\hat{z}}{\sqrt{5}}
  \end{array}  
$
\bigbreak

\noindent
$\displaystyle
  \begin{pmatrix}
    \dfrac{1}{\sqrt{6}} & \dfrac{1}{\sqrt{6}} & \dfrac{2}{\sqrt{6}}  \\
    \dfrac{1}{\sqrt{30}} & \dfrac{-5}{\sqrt{30}} & \dfrac{2}{\sqrt{30}} \\ 
    \dfrac{2}{\sqrt{5}} & 0 & \dfrac{-1}{\sqrt{5}} \\ 
  \end{pmatrix}
  =
  \begin{pmatrix}
    1 & 1 & 2\\
    1 & -5 & 2 \\ 
    2 & 0 & -1 \\ 
  \end{pmatrix}
  = 
  \begin{pmatrix}
    1 & 1 & 2\\
    2 & 0 & -1\\ 
    1 & -5 & 2\\ 
  \end{pmatrix}
  = 
  \begin{pmatrix}
    1 & 0 & 0\\
    0 & 1 & 0\\ 
    0 & 0 & 1\\ 
  \end{pmatrix}
  =
  I_3\times_3
$

\bigbreak


$
\overrightarrow{a}\times(\overrightarrow{b}\times\overrightarrow{c})=0
$

$
\overrightarrow{b}\times\overrightarrow{c}
=
\left|
\begin{matrix}
  \hat{x} & \hat{y} & \hat{z} \\
  \dfrac{1}{\sqrt{30}} & \dfrac{-5}{\sqrt{30}} & \dfrac{2}{\sqrt{30}} \\ 
  \dfrac{2}{\sqrt{5}} & 0 & \dfrac{-1}{\sqrt{5}} \\ 
\end{matrix}
\right|
=
\left|
\begin{matrix}
  \hat{x} & \hat{y} & \hat{z} \\
  1 & -5 & 2 \\ 
  2 & 0 & -1 \\ 
\end{matrix}
\right|
=
\hat{x}(5-0)-\hat{y}(-1-4)+\hat{z}(0-(-10))
$

$
=
5\hat{x}+5\hat{y}+10\hat{z}
=\overrightarrow{W}
$


$
\overrightarrow{a}\times\overrightarrow{W}
=
\left|
\begin{matrix}
  \hat{x} & \hat{y} & \hat{z} \\
  \dfrac{1}{\sqrt{6}} & \dfrac{1}{\sqrt{6}} & \dfrac{2}{\sqrt{6}}  \\ 
  5 & 5 & 10 \\ 
\end{matrix}
\right|
=
\hat{x}(10-10)-\hat{y}(10-10)+\hat{z}(5-5)
=
0
$

Showing the vectors $\mathbf{\hat{x}}, \mathbf{\hat{y}}, \mathbf{\hat{z}}$ in the basis of the vectors $\mathbf{a}, \mathbf{b}, \mathbf{c}$. An augmented matrix and row-reduction:

\makeatletter
\renewcommand*\env@matrix[1][*\c@MaxMatrixCols c]{
  \hskip -\arraycolsep
  \let\@ifnextchar\new@ifnextchar
  \array{#1}}
\makeatother
\begin{equation}
  \begin{bmatrix}[ccc|c]
    \dfrac{1}{\sqrt{6}} & \dfrac{1}{\sqrt{6}} & \dfrac{2}{\sqrt{6}} & \overrightarrow{a} \\
    \dfrac{1}{\sqrt{30}} & \dfrac{-5}{\sqrt{30}} & \dfrac{2}{\sqrt{30}} & \overrightarrow{b} \\
    \dfrac{2}{\sqrt{5}} & 0 & \dfrac{-1}{\sqrt{5}}  & \overrightarrow{c} \\
  \end{bmatrix}
  =
  \begin{bmatrix}[ccc|c]
    1 & 0 & 0 & \dfrac{\overrightarrow{a}}{6}+\dfrac{\overrightarrow{b}}{30}+\dfrac{2\overrightarrow{c}}{5} \\
    0 & 1 & 0 & \dfrac{\overrightarrow{a}}{6}-\dfrac{\overrightarrow{b}}{6} \\
    0 & 0 & 1 & \dfrac{\overrightarrow{a}}{3}+\dfrac{\overrightarrow{b}}{15}-\dfrac{\overrightarrow{c}}{5} \\
  \end{bmatrix}
\end{equation}


$\Rightarrow$ 
$
\begin{array}{lll}
  \hat{x}= \dfrac{\overrightarrow{a}}{6}+\dfrac{\overrightarrow{b}}{30}+\dfrac{2\overrightarrow{c}}{5} \\
  \hat{y}= \dfrac{\overrightarrow{a}}{6}-\dfrac{\overrightarrow{b}}{6} \\
  \hat{z}= \dfrac{\overrightarrow{a}}{3}+\dfrac{\overrightarrow{b}}{15}-\dfrac{\overrightarrow{c}}{5}
\end{array}  
$


$\Rightarrow$ 
$
\begin{array}{lll}
  A = \hat{x}+\hat{y}+2\hat{z} \\
  B = \hat{x}-\hat{y}+2\hat{z}\\
  C = \hat{y}-\hat{z}
\end{array}  
$

\bigbreak

$\Rightarrow$ 
$
\begin{array}{lll}
  A = \dfrac{2\overrightarrow{a}}{3}-\dfrac{\overrightarrow{b}}{15}+\dfrac{\overrightarrow{c}}{5} \\
  B = \dfrac{2\overrightarrow{a}}{3}+\dfrac{\overrightarrow{b}}{3} \\
  C = \dfrac{-\overrightarrow{a}}{6}-\dfrac{7\overrightarrow{b}}{30}+\dfrac{\overrightarrow{c}}{5}
\end{array}  
$


\item Mini-quiz: which of these sets of vectors is not a basis for a three dimensional cartesian space? 
\begin{itemize}
\item $\{ \mathbf{\hat{x}}, \mathbf{\hat{x}}+3\mathbf{\hat{y}}, 3\mathbf{\hat{x}}-\mathbf{\hat{y}} \}$

\noindent
$\displaystyle
  \begin{pmatrix}
    1 & 0 & 0 \\
    1 & 3 & 0 \\ 
    3 & -1 & 0 \\ 
  \end{pmatrix}
  =
  \begin{pmatrix}
    1 & 0 & 0 \\
    0 & 1 & 0 \\ 
    0 & 0 & 0 \\ 
  \end{pmatrix}
$

Therefore, the set of vectors does not form a basis for $R^3$.


\item $\{ \mathbf{\hat{z}}, \mathbf{\hat{x}}+3\mathbf{\hat{y}}, 3\mathbf{\hat{x}}-\mathbf{\hat{y}}\}$

\noindent
$\displaystyle
  \begin{pmatrix}
    0 & 0 & 1 \\
    1 & 3 & 0 \\ 
    3 & -1 & 0 \\ 
  \end{pmatrix}
  =
  \begin{pmatrix}
    1 & 0 & 0 \\
    0 & 1 & 0 \\ 
    0 & 0 & 1 \\ 
  \end{pmatrix}
$

Therefore, the set of vectors forms a basis for $R^3$.

\item $\{ e^\pi \mathbf{\hat{z}}, \mathbf{\hat{x}}+3\mathbf{\hat{y}}, 3\mathbf{\hat{x}}-\mathbf{\hat{y}}\}$

\noindent
$\displaystyle
  \begin{pmatrix}
    0 & 0 & e^\pi \\
    1 & 3 & 0 \\ 
    3 & -1 & 0 \\ 
  \end{pmatrix}
  =
  \begin{pmatrix}
    1 & 0 & 0 \\
    0 & 1 & 0 \\ 
    0 & 0 & 1 \\ 
  \end{pmatrix}
$

Therefore, the set of vectors forms a basis for $R^3$.
\end{itemize}


\item  Compute the sum and difference of the following two matrices:
$$A=\begin{pmatrix}
2 & 5 & -1 \\
-3 & 4 & 2 \\
1 & 7 & 3
\end{pmatrix} \hskip 1truecm B=\begin{pmatrix}
5 & -5 & 3 \\
1 & 4 & 3 \\
-4 & 2 & 1
\end{pmatrix}$$ 

Note: $A+B=B+A$
\\
\noindent
$A+B=
\begin{pmatrix}
    2 & 5 & -1 \\
    -3 & 4 & 2 \\ 
    1 & 7 & 3 \\ 
\end{pmatrix}
+
\begin{pmatrix}
  5 & -5 & 3 \\
  1 & 4 & 3 \\
  -4 & 2 & 1
\end{pmatrix}
=
\begin{pmatrix}
  2+5 & 5+(-5) & -1+3 \\
  -3+1 & 4+4 & 2+3 \\
  1+(-4) & 7+2 & 3+1
\end{pmatrix}
=
\begin{pmatrix}
  7 & 0 & 2 \\
  -2 & 8 & 5 \\
  -3 & 9 & 4
\end{pmatrix}
$

\noindent
$A-B=
\begin{pmatrix}
    2 & 5 & -1 \\
    -3 & 4 & 2 \\ 
    1 & 7 & 3 \\ 
\end{pmatrix}
-
\begin{pmatrix}
  5 & -5 & 3 \\
  1 & 4 & 3 \\
  -4 & 2 & 1
\end{pmatrix}
=
\begin{pmatrix}
  2-5 & 5-(-5) & -1-3 \\
  -3-1 & 4-4 & 2-3 \\
  1-(-4) & 7-2 & 3-1
\end{pmatrix}
=
\begin{pmatrix}
  -3 & 10 & -4 \\
  -4 & 0 & -1 \\
  5 & 5 & 2
\end{pmatrix}
$

\noindent
$B-A=
\begin{pmatrix}
  5 & -5 & 3 \\
  1 & 4 & 3 \\
  -4 & 2 & 1
\end{pmatrix}
-
\begin{pmatrix}
  2 & 5 & -1 \\
  -3 & 4 & 2 \\ 
  1 & 7 & 3 \\ 
\end{pmatrix}
=
\begin{pmatrix}
  5-2 & -5-5 & 3-(-1) \\
  1-(-3) & 4-4 & 3-2 \\
  -4-1 & 2-7 & 1-3
\end{pmatrix}
=
\begin{pmatrix}
  3 & -10 & 4 \\
  4 & 0 & 1 \\
  -5 & -5 & -2
\end{pmatrix}
$



\end{enumerate}

\end{document}
