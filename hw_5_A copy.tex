\documentclass[fleqn]{article}
\oddsidemargin 0.0in
\textwidth 6.0in
\thispagestyle{empty}
\usepackage{import}
\usepackage{amsmath}
\usepackage{graphicx}


\begin{document}

%useful notation: \mathbf{A}=\mathbf{\hat{x}}+2\mathbf{\hat{y}-}2\mathbf{\hat{z}, \quad \quad B}=3\mathbf{\hat{x}}+\mathbf{\hat{y}+}2\mathbf{\hat{z}, \quad \quad C}=4\mathbf{\hat{x}}-\mathbf{\hat{y}+\hat{z}.}

%matrix template: A = 
%\begin{pmatrix}
%1 & 2 & 3 \\
%4 & 5 & 6 \\
%7 & 8 & 9
%\end{pmatrix}

{\bf Homework 5 - Part A}\\
Homework assigned on a given day is always due on the following Tuesday at noon. Electronic submission via Canvas is required. Single file, .pdf or .jpeg only.  A scannerized copy of a handwritten file is acceptable. Typeset solutions are welcome as well. The student is responsible for the readability of the file. 

\begin{enumerate}


\item Consider the vectors: $\mathbf{a}=( \mathbf{\hat{x}} - \mathbf{\hat{y}} ) / \sqrt{2} $, $\mathbf{b}=( \mathbf{\hat{x}} + \mathbf{\hat{y}} + 2 \mathbf{\hat{z}} ) / \sqrt{6}$ and $\mathbf{c}=(\mathbf{\hat{x}}+\mathbf{\hat{y}}-\mathbf{\hat{z}})/\sqrt{3}$. 
% . 
\begin{enumerate}
\item Prove that they are a basis for a three dimensional cartesian space, and check if they are an \emph{orthogonal} basis. 


\[
\begin{array}{lll}
a.\frac{\hat{x}-\hat{y}}{\sqrt{2}} \\
b.\frac{\hat{x}+\hat{y}+2\hat{z}}{\sqrt{6}} \\
c.\frac{\hat{x}+\hat{y}-\hat{z}}{\sqrt{3}}
\end{array}
\]








\item Express the vectors $\mathbf{\hat{x}}, \mathbf{\hat{y}}, \mathbf{\hat{z}}$ in the basis of the vectors $\mathbf{a}, \mathbf{b}, \mathbf{c}$. 

\item Consider now the set of vectors $\mathbf{A}=\mathbf{\hat{x}}-\mathbf{\hat{y}}$; $\mathbf{B}=\mathbf{\hat{x}}+\mathbf{\hat{z}}$; $\mathbf{C}=\mathbf{\hat{y}}-2\mathbf{\hat{z}}$, and express them in the basis of the vectors $\mathbf{a}, \mathbf{b}, \mathbf{c}$. 

\end{enumerate}


\item Same as the previous exercise, with $\mathbf{a}=\frac{1}{\sqrt{6}} ( \mathbf{\hat{x}} + \mathbf{\hat{y}} + 2 \mathbf{\hat{z}} ) $, $\mathbf{b}=\frac{1}{\sqrt{30}} (\mathbf{\hat{x}} - 5\mathbf{\hat{y}} + 2\mathbf{\hat{z}})$ and $\mathbf{c}= \frac{1}{\sqrt{5}} ( 2 \mathbf{\hat{x} }- \mathbf{\hat{z} })$, and 
$A=\mathbf{\hat{x}}+\mathbf{\hat{y}}+2\mathbf{\hat{z}}; B=\mathbf{\hat{x}}-\mathbf{\hat{y}}+2\mathbf{\hat{z}}; C=\mathbf{\hat{y}}-\mathbf{\hat{z}}$. 


\item Mini-quiz: which of these sets of vectors is not a basis for a three dimensional cartesian space? 
\begin{itemize}
\item $\{ \mathbf{\hat{x}}, \mathbf{\hat{x}}+3\mathbf{\hat{y}}, 3\mathbf{\hat{x}}-\mathbf{\hat{y}} \}$

\item $\{ \mathbf{\hat{z}}, \mathbf{\hat{x}}+3\mathbf{\hat{y}}, 3\mathbf{\hat{x}}-\mathbf{\hat{y}}\}$

\item $\{ e^\pi \mathbf{\hat{z}}, \mathbf{\hat{x}}+3\mathbf{\hat{y}}, 3\mathbf{\hat{x}}-\mathbf{\hat{y}}\}$
\end{itemize}


\item  Compute the sum and difference of the following two matrices:
$$A=\begin{pmatrix}
2 & 5 & -1 \\
-3 & 4 & 2 \\
1 & 7 & 3
\end{pmatrix} \hskip 1truecm B=\begin{pmatrix}
5 & -5 & 3 \\
1 & 4 & 3 \\
-4 & 2 & 1
\end{pmatrix}$$ 



\end{enumerate}

\end{document}
