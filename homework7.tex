\documentclass[fleqn]{article}
\oddsidemargin 0.0in
\textwidth 6.0in
\thispagestyle{empty}
\usepackage{import}
\usepackage{amsmath}
\usepackage{graphicx}
\usepackage[english]{babel}
\usepackage[utf8x]{inputenc}
\usepackage{float}
\usepackage[colorinlistoftodos]{todonotes}

\definecolor{hwColor}{HTML}{AD53BA}

\begin{document}

  \begin{titlepage}

    \newcommand{\HRule}{\rule{\linewidth}{0.5mm}} % Defines a new command for the horizontal lines, change thickness here

    \center % Center everything on the page
    


    \textsc{\LARGE Arizona State University}\\[1.5cm] % Name of your university/college

    \textsc{\LARGE Mathematical Methods For Physics I }\\[1.5cm] % Major heading such as course name


    \begin{figure}
      \includegraphics[width=\linewidth]{asu.png}
    \end{figure}


    \HRule \\[0.4cm]
    { \huge \bfseries Homework 7}\\[0.4cm] 
    \HRule \\[1.5cm]
    
    \textbf{Behnam Amiri}

    \bigbreak

    \textbf{Prof: Cecilia Lunardini}

    \bigbreak


    \textbf{{\large \today}\\[2cm]}

    \vfill % Fill the rest of the page with whitespace

  \end{titlepage}

  \begin{enumerate}
    \item Calculate the eigenvalues and eigenvectors of the Pauli matrices.

      \textcolor{hwColor}{ 
        To find the eigenvectors of a matrix called $\sigma$, we will need to find the eigenvalues, and to find the eigenvalues, we will need to start by finding the determinant of the matrix $\sigma-\lambda I$. From 
        From equation 8.86 in the textbook, we know that we need to set the determinant of $\sigma-\lambda I$ to zeroin order to find the eigenvalues. 
      }

      \textcolor{hwColor}{ 
        $\sigma_1=\sigma_x\equiv
        \begin{pmatrix}
          0 & 1 \\
          1 & 0 
        \end{pmatrix}
         , \sigma_2=\sigma_y\equiv
        \begin{pmatrix}
          0 & -i \\
          i & 0 
        \end{pmatrix}
        , \sigma_3=\sigma_z\equiv
        \begin{pmatrix}
          1 & 0 \\
          0 & -1 
        \end{pmatrix}
        $
      }

      \bigbreak

      \textcolor{hwColor}{ 
        $\sigma_1-\lambda I=
          \begin{pmatrix}
            0 & 1 \\
            1 & 0 
          \end{pmatrix}-
          \lambda \begin{pmatrix}
            1 & 0 \\
            0 & 1
          \end{pmatrix}=
          \begin{pmatrix}
            0 & 1 \\
            1 & 0 
          \end{pmatrix}-
          \lambda \begin{pmatrix}
            \lambda & 0 \\
            0 & \lambda
          \end{pmatrix}=
          \begin{pmatrix}
            -\lambda & 1 \\
            1 & -\lambda
          \end{pmatrix}
        $
      }

      \textcolor{hwColor}{ 
        $det(\sigma_1-\lambda I)=
          \begin{vmatrix}
            -\lambda & 1 \\
            1 & -\lambda
          \end{vmatrix}= \lambda^2-1
        $
        if $det(\sigma_1-\lambda I)=0$ 
        $
          \Longrightarrow
          \begin{cases}
            \lambda_1=1 \\
            \lambda_2=-1
          \end{cases}
        $
      }

      \textcolor{hwColor}{ $\sigma_1 \overrightarrow{x}=\lambda \overrightarrow{x}$ }

      \textcolor{hwColor}{ 
       $
       \lambda_1=1  \rightarrow 
       \begin{pmatrix}
        0 & 1 \\
        1 & 0  
       \end{pmatrix}.\begin{pmatrix}
         x \\
         y
       \end{pmatrix}=\begin{pmatrix}
         x \\
         y
       \end{pmatrix}
       \Longrightarrow 
       \begin{cases}
         y=x \\
         x=y
       \end{cases}
       $
       if $x=1$ then $y=1$, therefore $v_1=\begin{pmatrix}
         1 \\
         1
       \end{pmatrix}$
      }

      \textcolor{hwColor}{ 
        $
        \lambda_2=-1  \rightarrow 
        \begin{pmatrix}
         0 & 1 \\
         1 & 0  
        \end{pmatrix}.\begin{pmatrix}
          x \\
          y
        \end{pmatrix}=\begin{pmatrix}
          -x \\
          -y
        \end{pmatrix}
        \Longrightarrow 
        \begin{cases}
          y=-x \\
          x=-y
        \end{cases}
        $
        if $y=1$ then $x=-1$, therefore $v_2=\begin{pmatrix}
          -1 \\
          1
        \end{pmatrix}$
       }

      \textcolor{hwColor}{
        \rule{16cm}{0.4pt}
      }

      \textcolor{hwColor}{ 
        $\sigma_2-\lambda I=
          \begin{pmatrix}
            0 & -i \\
            i & 0 
          \end{pmatrix}-
          \lambda \begin{pmatrix}
            1 & 0 \\
            0 & 1
          \end{pmatrix}=
          \begin{pmatrix}
            0 & -i \\
            i & 0 
          \end{pmatrix}-
          \begin{pmatrix}
            \lambda & 0 \\
            0 & \lambda
          \end{pmatrix}=
          \begin{pmatrix}
            -\lambda & -i \\
            i & -\lambda 
          \end{pmatrix}
        $
      }

      \textcolor{hwColor}{ 
        $det(\sigma_2-\lambda I)=
          \begin{vmatrix}
            -\lambda & -i \\
            i & -\lambda
          \end{vmatrix}= \lambda^2-(-i^2)=\lambda^2-1
        $
        if $det(\sigma_2-\lambda I)=0$ 
        $
          \Longrightarrow
          \begin{cases}
            \lambda_1=1 \\
            \lambda_2=-1
          \end{cases}
        $
      }

      \textcolor{hwColor}{ $\sigma_2 \overrightarrow{x}=\lambda \overrightarrow{x}$ }

      \textcolor{hwColor}{ 
        $
        \lambda_1=1  \rightarrow 
        \begin{pmatrix}
         0 & -i \\
         i & 0  
        \end{pmatrix}.\begin{pmatrix}
          x \\
          y
        \end{pmatrix}=\begin{pmatrix}
          x \\
          y
        \end{pmatrix}
        \Longrightarrow 
        \begin{cases}
          -iy=x \\
          ix=y
        \end{cases}
        $
        if $y=1$ then $x=i$, therefore $v_1=\begin{pmatrix}
          i \\
          1
        \end{pmatrix}$
       }

       \textcolor{hwColor}{ 
        $
        \lambda_1=-1  \rightarrow 
        \begin{pmatrix}
         0 & -i \\
         i & 0  
        \end{pmatrix}.\begin{pmatrix}
          x \\
          y
        \end{pmatrix}=\begin{pmatrix}
          -x \\
          -y
        \end{pmatrix}
        \Longrightarrow 
        \begin{cases}
          -iy=-x \\
          ix=-y
        \end{cases}
        $
        if $y=1$ then $x=-i$, therefore $v_2=\begin{pmatrix}
          -i \\
          1
        \end{pmatrix}$
       }

      \textcolor{hwColor}{
        \rule{16cm}{0.4pt}
      }

      \textcolor{hwColor}{ 
        $\sigma3-\lambda I=
          \begin{pmatrix}
            1 & 0 \\
            0 & -1 
          \end{pmatrix}-
          \lambda \begin{pmatrix}
            1 & 0 \\
            0 & 1
          \end{pmatrix}=
          \begin{pmatrix}
            1 & 0 \\
            0 & -1 
          \end{pmatrix}-
          \begin{pmatrix}
            \lambda & 0 \\
            0 & \lambda
          \end{pmatrix}=
          \begin{pmatrix}
            1-\lambda & 0 \\
            0 & -1-\lambda 
          \end{pmatrix}
        $
      }

      \textcolor{hwColor}{ 
        $det(\sigma_3-\lambda I)=
          \begin{vmatrix}
            1-\lambda & 0 \\
            0 & -1-\lambda 
          \end{vmatrix}= (1-\lambda)(-1-\lambda)
        $
        if $det(\sigma_3-\lambda I)=0$ 
        $
          \Longrightarrow
          \begin{cases}
            \lambda_1=1 \\
            \lambda_2=-1
          \end{cases}
        $
      }

      \textcolor{hwColor}{ $\sigma_3 \overrightarrow{x}=\lambda \overrightarrow{x}$ }

      \textcolor{hwColor}{ 
        $
        \lambda_1=1  \rightarrow 
        \begin{pmatrix}
          1 & 0 \\
          0 & -1 
        \end{pmatrix}.\begin{pmatrix}
          x \\
          y
        \end{pmatrix}=\begin{pmatrix}
          x \\
          y
        \end{pmatrix}
        \Longrightarrow 
        \begin{cases}
          x=x \\
          -y=y
        \end{cases}
        $
        $x=1$ and $y=0$, therefore $v_1=\begin{pmatrix}
          1 \\
          0
        \end{pmatrix}$
      }

      
      \textcolor{hwColor}{ 
        $
        \lambda_1=-1  \rightarrow 
        \begin{pmatrix}
         0 & -i \\
         i & 0  
        \end{pmatrix}.\begin{pmatrix}
          x \\
          y
        \end{pmatrix}=\begin{pmatrix}
          -x \\
          -y
        \end{pmatrix}
        \Longrightarrow 
        \begin{cases}
          x=-x \\
          y=-y
        \end{cases}
        $
        $x=0$ and $y=1$, therefore $v_2=\begin{pmatrix}
          0 \\
          1
        \end{pmatrix}$
      }


    \item Show that if a matrix is diagonal, (i.e., its only non-zero elements lie on the main (upper-left to lower-right) diagonal), these elements are the matrix's eigenvalues. What are the corresponding eigenvectors?

      \textcolor{hwColor}{
        A diagonal matrix is a matrix in which the entries outside the main diagonal are all zero.
        $
        A=\begin{pmatrix}
          a & 0 & 0 \\
          0 & b & 0 \\
          0 & 0 & c \\
        \end{pmatrix}
        $
      }

      \textcolor{hwColor}{
        $
        A-\lambda I=\begin{pmatrix}
          a & 0 & 0 \\
          0 & b & 0 \\
          0 & 0 & c \\
        \end{pmatrix}-\begin{pmatrix}
          \lambda & 0 & 0 \\
          0 & \lambda & 0 \\
          0 & 0 & \lambda \\
        \end{pmatrix}=\begin{pmatrix}
          a-\lambda & 0 & 0 \\
          0 & b-\lambda & 0 \\
          0 & 0 & c-\lambda \\
        \end{pmatrix}
        $
      }

      \textcolor{hwColor}{
        $
        det(A-\lambda I)=\begin{vmatrix}
          a-\lambda & 0 & 0 \\
          0 & b-\lambda & 0 \\
          0 & 0 & c-\lambda \\
        \end{vmatrix}=(a-\lambda)\begin{vmatrix}
          b-\lambda & 0 \\
          0 & c-\lambda \\
        \end{vmatrix}=(a-\lambda)(b-\lambda)(c-\lambda)
        $
      }

      \textcolor{hwColor}{
        If $det(A-\lambda I)=0$
        , then $=(a-\lambda)(b-\lambda)(c-\lambda)=0$
        $
        \begin{cases}
          \lambda_1=a \\
          \lambda_2=b \\
          \lambda_3=c \\
        \end{cases}
        $
      }

      \textcolor{hwColor}{
        $
        \lambda_1=a \rightarrow 
        \begin{pmatrix}
          a & 0 & 0 \\
          0 & b & 0 \\
          0 & 0 & c \\
        \end{pmatrix}.\begin{pmatrix}
          x \\
          y \\
          z \\
        \end{pmatrix}=a\begin{pmatrix}
          x \\
          y \\
          z \\
        \end{pmatrix} 
        \Longrightarrow 
        \begin{cases}
          ax=ax \\
          by=ay \\
          cz=az \\
        \end{cases}
        $
      }

      \textcolor{hwColor}{
        $
        \Longrightarrow 
        x=1, y=0$, and $z=0$, hence $v_1=\begin{pmatrix}
          1 \\
          0 \\
          0 \\
        \end{pmatrix}
        $
      }

      \textcolor{hwColor}{
        \rule{16cm}{0.4pt}
      }

      \textcolor{hwColor}{
        $
        \lambda_2=b \rightarrow 
        \begin{pmatrix}
          a & 0 & 0 \\
          0 & b & 0 \\
          0 & 0 & c \\
        \end{pmatrix}.\begin{pmatrix}
          x \\
          y \\
          z \\
        \end{pmatrix}=b\begin{pmatrix}
          x \\
          y \\
          z \\
        \end{pmatrix} 
        \Longrightarrow 
        \begin{cases}
          ax=bx \\
          by=by \\
          cz=bz \\
        \end{cases}
        $
      }

      \textcolor{hwColor}{
        $
        \Longrightarrow 
        x=0, y=1$, and $z=0$, hence $v_2=\begin{pmatrix}
          0 \\
          1 \\
          0 \\
        \end{pmatrix}
        $
      }

      
      \textcolor{hwColor}{
        \rule{16cm}{0.4pt}
      }

      \textcolor{hwColor}{
        $
        \lambda_3=c \rightarrow 
        \begin{pmatrix}
          a & 0 & 0 \\
          0 & b & 0 \\
          0 & 0 & c \\
        \end{pmatrix}.\begin{pmatrix}
          x \\
          y \\
          z \\
        \end{pmatrix}=c\begin{pmatrix}
          x \\
          y \\
          z \\
        \end{pmatrix} 
        \Longrightarrow 
        \begin{cases}
          ax=cx \\
          by=cy \\
          cz=cz \\
        \end{cases}
        $
      }

      \textcolor{hwColor}{
        $
        \Longrightarrow 
        x=0, y=0$, and $z=1$, hence $v_3=\begin{pmatrix}
          0 \\
          0 \\
          1 \\
        \end{pmatrix}
        $
      }

  
    \item The matrix 
      $$M = 
      \begin{pmatrix}
      1 & -1 & -1 \\
      -1 & 1 & -1 \\
      -1 & -1 & 1
      \end{pmatrix}
      $$
      has a 2-fold degenerate eigenvalue. Find the eigenvalues and construct a full set of 3 different
      (linearly independent) eigenvectors. Normalize the eigenvectors.

      \textcolor{hwColor}{
        $
          M-\lambda I= \begin{pmatrix}
            1 & -1 & -1 \\
            -1 & 1 & -1 \\
            -1 & -1 & 1
          \end{pmatrix}-\begin{pmatrix}
            \lambda & 0 & 0 \\
            0 & \lambda & 0 \\
            0 & 0 & \lambda
          \end{pmatrix}=\begin{pmatrix}
            1-\lambda & -1 & -1 \\
            -1 & 1-\lambda & -1 \\
            -1 & -1 & 1-\lambda
          \end{pmatrix}
        $
      }

      \textcolor{hwColor}{
        $det(M-\lambda I)=(1-\lambda)\left[(1-\lambda)(1-\lambda)-1\right]+\left[(\lambda-1)-1\right]-\left[1-(\lambda-1)\right]=-\lambda^3+3\lambda^2-4$
      }

      \textcolor{hwColor}{
        If $det(M-\lambda I)=0$, then $-\lambda^3+3\lambda^2-4=0 \rightarrow \lambda^3-3\lambda^2+4=0$
      }

      \textcolor{hwColor}{
        $
        \rightarrow \lambda^3-3\lambda^2+4=(\lambda^2-\lambda-2)(2-\lambda)=(\lambda+1)(\lambda-2)(2-\lambda)=0
        \Longrightarrow 
        \begin{cases}
          \lambda_1=-1 \\
          \lambda_2=2 \\
          \lambda_3=2 \\
        \end{cases}
        $
      }

      \textcolor{hwColor}{
        $
        \lambda_1=-1 \rightarrow
        \begin{pmatrix}
          1 & -1 & -1 \\
          -1 & 1 & -1 \\
          -1 & -1 & 1
        \end{pmatrix}.\begin{pmatrix}
          x \\
          y \\
          z \\
        \end{pmatrix}=\begin{pmatrix}
          -x \\
          -y \\
          -z \\
        \end{pmatrix}
        \Longrightarrow
        \begin{cases}
          x-y-z=-x \\
          -x+y-z=-y \\
          -x-y+z=-z
        \end{cases}
        $
      }

      \textcolor{hwColor}{
        After a little algebra, we find that the only way to satisfy all three equations simultaneously is if $x=y=z$. It does not matter what value they are, only that they all have the same value. For simplicity, we pick $x=y=z=1$.
        Hence, first eigenvector is 
        $
        V_1=\begin{pmatrix}
          1 \\
          1 \\
          1
        \end{pmatrix}
        $
        which, after normalization, becomes, $V_1=\dfrac{1}{\sqrt{3}} \begin{pmatrix}
          1 \\
          1 \\
          1 \\
        \end{pmatrix}
        $
      }

      \textcolor{hwColor}{
        \rule{16cm}{0.4pt}
      }

      \bigbreak

      \textcolor{hwColor}{
        $
        \lambda_2=\lambda_3=-1 \rightarrow
        \begin{pmatrix}
          1 & -1 & -1 \\
          -1 & 1 & -1 \\
          -1 & -1 & 1
        \end{pmatrix}.\begin{pmatrix}
          x \\
          y \\
          z \\
        \end{pmatrix}=\begin{pmatrix}
          2x \\
          2y \\
          2z \\
        \end{pmatrix}
        \Longrightarrow
        \begin{cases}
          x-y-z=2x \\
          -x+y-z=2y \\
          -x-y+z=2z
        \end{cases}
        $
      }

      \textcolor{hwColor}{
        After a little algebra, it turns out they all end up being the same $-x-y-z=0$. We can rearrange it to $x=-(y+z)$. Now whatever we end up with for $V_2$ and $V_3$,
        we will have to ensure that one is not a scalar multiple of the other. Since it appears that we have no constraints on the values of $y$ and $z$ at this point, for simplicity let's set $y=0$.
        Doing so gives us $x=-z$. Since there is no constraints on the value of $z$, we shall pick $x=1$. Therefore,       
        $
        V_2=\begin{pmatrix}
          1 \\
          0 \\
          -1
        \end{pmatrix}
        $
        which, after normalization, becomes, $V_2=\dfrac{1}{\sqrt{2}} \begin{pmatrix}
          1 \\
          0 \\
          -1
        \end{pmatrix}
        $
      }

      \textcolor{hwColor}{
        Now let's consider when we set $z=0$. Then $x=-y$, and if we choose $x=1$, we end up with the following vector.       
        $
        V_3=\begin{pmatrix}
          1 \\
          -1 \\
          0
        \end{pmatrix}
        $
        which, after normalization, becomes, $V_3=\dfrac{1}{\sqrt{2}} \begin{pmatrix}
          1 \\
          -1 \\
          0
        \end{pmatrix}
        $
      }

    \item The matrix in the previous exercise is Hermitian. If the degenerate eigenvectors you found are not orthogonal, find (using any method of your choice) two that are, and normalize them so that your set of eigenvectors is orthonormal.
      Compute the right hand side of eq. (8.80) of the textbook as a check of your result (what are you supposed to find? Comment as adequate).

      \textcolor{hwColor}{
        $v_2.v_3=(1,0,-1).(1,-1,0)=1+0+0=1$ \\
        We can stop right there, since the scalar product of the two vectors is NOT zero, the two eigenvectors I found are not orthogonal. 
      }

      \bigbreak
      
      \textcolor{hwColor}{
        \textbf{Gram-Schmidt:} \\
        $\overrightarrow{x_1}=\overrightarrow{v_1}=\begin{pmatrix}
          1 \\
          1 \\
          1 \\
        \end{pmatrix}$ \\
        $
          \overrightarrow{x_2}=\overrightarrow{v_2}-\dfrac{\overrightarrow{v_2}.\overrightarrow{v_1}}{\overrightarrow{v_1}.\overrightarrow{v_1}}.\overrightarrow{v_1}
          =\begin{pmatrix}
            1 \\
            0 \\
            -1 \\
          \end{pmatrix}-\dfrac{(1,0,-1).(1,1,1)}{(1,1,1).(1,1,1)}.(1,1,1) \Longrightarrow \overrightarrow{x_2}=\begin{pmatrix}
            1 \\
            0 \\
            -1 \\
          \end{pmatrix}
        $ \\
        $
        \overrightarrow{x_3}=\overrightarrow{v_3}\left[\dfrac{\overrightarrow{v_3}.\overrightarrow{v_1}}{\overrightarrow{v_1}\overrightarrow{v_1}}\right]-\left[\dfrac{\overrightarrow{v_3}.\overrightarrow{v_2}}{\overrightarrow{v_2}.\overrightarrow{v_2}}\right]\overrightarrow{v_2}=\begin{pmatrix}
          1 \\
          -1 \\
          0
        \end{pmatrix}-\dfrac{(1,-1,0).(1,1,1)}{(1,1,1).(1,1,1)}.(1,1,1) \Longrightarrow \overrightarrow{x_3}=\begin{pmatrix}
          \dfrac{1}{2} \\
          0 \\
          -\dfrac{1}{2} \\
        \end{pmatrix}
        $
      }

      \textcolor{hwColor}{
        Eq 8.80
      }
      \begin{equation}
        \textcolor{hwColor}{
          \sum^{N}_{i=1} \lambda_i x^i(x^i)^\dagger 
         }
      \end{equation}


      \textcolor{hwColor}{
        Now I we just need to substitue in the vectors and eigenvalues we got into the equation.
        $
        -1\begin{pmatrix}
          \dfrac{1}{\sqrt{3}} \\
          \dfrac{1}{\sqrt{3}} \\
          \dfrac{1}{\sqrt{3}} \\
        \end{pmatrix}
        $  and \\
        $
          2\begin{pmatrix}
            \dfrac{1}{\sqrt{2}} \\
            0 \\
            -\dfrac{1}{\sqrt{2}} \\
          \end{pmatrix}
        $
      }
      

    \item Find the eigenvalues and a set of eigenvectors of the matrix
      $$ 
      \begin{pmatrix}
        1 & 3 & -1 \\
        3 & 4 & -2 \\
        -1 & -2 & 2 \\
      \end{pmatrix}
      $$
      Verify that its eigenvectors are mutually orthogonal.

      \textcolor{hwColor}{
        Let's call the above matrix Q.
      }

      \textcolor{hwColor}{
        $
          Q-\lambda I= \begin{pmatrix}
            1 & 3 & -1 \\
            3 & 4 & -2 \\
            -1 & -2 & 2 \\
          \end{pmatrix}-\begin{pmatrix}
            \lambda & 0 & 0 \\
            0 & \lambda & 0 \\
            0 & 0 & \lambda
          \end{pmatrix}=\begin{pmatrix}
            1-\lambda & 3 & -1 \\
            3 & 4-\lambda & -2 \\
            -1 & -2 & 2-\lambda
          \end{pmatrix}
        $
      }

      \textcolor{hwColor}{
        $det(Q-\lambda I)=-\lambda^3+7\lambda^2-6$
      }

      \textcolor{hwColor}{
        If $det(Q-\lambda I)=0$, then $-\lambda^3+7\lambda^2-6=0$
      }

      \textcolor{hwColor}{
        $
        \Longrightarrow 
        \begin{cases}
          \lambda_1=1 \\
          \lambda_2=3+\sqrt{15} \\
          \lambda_3=3-\sqrt{15} \\
        \end{cases}
        $
      }

      \textcolor{hwColor}{
        $
        \lambda_1=1 \rightarrow
        \begin{pmatrix}
          1 & 3 & -1 \\
          3 & 4 & -2 \\
          -1 & -2 & 2 \\
        \end{pmatrix}.\begin{pmatrix}
          x \\
          y \\
          z \\
        \end{pmatrix}=\begin{pmatrix}
          x \\
          y \\
          z \\
        \end{pmatrix}
        \rightarrow
        \begin{cases}
          x+3y-z=x \\
          3x+4y-2z=y \\
          -x-2y+2z=z
        \end{cases}
        \Longrightarrow
        \begin{cases}
         3y-z=0  \\
         3x+3y-2z=0  \\
         -x-2y+z=0 
        \end{cases}
        $
      }

      \textcolor{hwColor}{
        $
          V_1=\begin{pmatrix}
            1 \\
            1 \\
            3
          \end{pmatrix}
        $
      }

      \textcolor{hwColor}{
        \rule{16cm}{0.4pt}
      }

      \bigbreak

      \textcolor{hwColor}{
        $
        \lambda_2=3+\sqrt{5} \rightarrow
        \begin{pmatrix}
          1 & 3 & -1 \\
          3 & 4 & -2 \\
          -1 & -2 & 2 \\
        \end{pmatrix}.\begin{pmatrix}
          x \\
          y \\
          z \\
        \end{pmatrix}=\begin{pmatrix}
          (3+\sqrt{15})x \\
          (3+\sqrt{15})y \\
          (3+\sqrt{15})z \\
        \end{pmatrix}
        \rightarrow
        \begin{cases}
          x+3y-z=(3+\sqrt{15})x \\
          3x+4y-2z=(3+\sqrt{15})y \\
          -x-2y+2z=(3+\sqrt{15})z
        \end{cases}
        \Longrightarrow
        \begin{cases}
         (-2- \sqrt{15})x+3y-z=0  \\
         3x+(1-\sqrt{15})y-2z=0  \\
         -x-2y+(-1-\sqrt{15})z=0 \\
        \end{cases}
        $ \\
        After doing some ao the above system of equations is $x=y=z=0$ and (0,0,0) can't be an eigenvector.
      }
      
    \item By finding the eigenvectors of the Hermitian matrix
      $$H= 
      \begin{pmatrix}
      10 & 3i \\
      -3i & 2 
      \end{pmatrix}
      $$

      Construct a unitary matrix $U$ such that $U^\dagger HU=\Lambda$, where $\Lambda$ is a real  diagonal matrix.
      
      \bigbreak

      \textcolor{hwColor}{
        The \emph{Hermitian} conjugate of a matrix is the \emph{transpose} of its \emph{complex conjugate}. \\
        $
          H^*=\begin{pmatrix}
            10 & -3i \\
            3i & 2 \\
          \end{pmatrix}
        $ and 
        $
          (H^*)^T=\begin{pmatrix}
            10 & 3i \\
            -3i & 2 \\
          \end{pmatrix}
          \Longrightarrow
          H^\dagger=\begin{pmatrix}
            10 & 3i \\
            -3i & 2 \\
          \end{pmatrix}
        $
      }

      \textcolor{hwColor}{
        $
          H-\lambda I=\begin{pmatrix}
            10 & -3i \\
            3i & 2 \\
          \end{pmatrix}-\begin{pmatrix}
            \lambda & 0 \\
            0 & \lambda \\
          \end{pmatrix}=\begin{pmatrix}
            10-\lambda & 3i \\
            -3i & 2-\lambda \\
          \end{pmatrix}
        $
      }

      \textcolor{hwColor}{
        $
         det(H-\lambda I)=\begin{vmatrix}
          10-\lambda & 3i \\
          -3i & 2-\lambda \\
         \end{vmatrix}=\lambda^2-12\lambda+11 \Longrightarrow 
         \begin{cases}
           \lambda_1=1 \\
           \lambda_2=11
         \end{cases}
        $
      }

      \textcolor{hwColor}{
        \rule{16cm}{0.4pt}
      }

      \textcolor{hwColor}{
        $
         \lambda_1=1 \rightarrow \begin{pmatrix}
          10 & -3i \\
          3i & 2 \\
         \end{pmatrix}.\begin{pmatrix}
           x \\
           y \\
         \end{pmatrix}=\begin{pmatrix}
          x \\
          y \\
        \end{pmatrix} \Longrightarrow 
        \begin{cases}
          9x+3yi=0 \\
          -3xi+y=0
        \end{cases} 
        $
        if $x=1$, then $y=3i$ \\
        $
          \Longrightarrow
          V_1=\begin{pmatrix}
            1 \\
            3i
          \end{pmatrix}
        $.
        After normalization $V_1$ becomes:
        $
        V_1=\dfrac{1}{\sqrt{10}}\begin{pmatrix}
          1 \\
          3i \\
        \end{pmatrix}
        $
      }

      \textcolor{hwColor}{
        $
         \lambda_2=11 \rightarrow \begin{pmatrix}
          10 & -3i \\
          3i & 2 \\
         \end{pmatrix}.\begin{pmatrix}
           x \\
           y \\
         \end{pmatrix}=\begin{pmatrix}
          11x \\
          11y \\
        \end{pmatrix} \Longrightarrow 
        \begin{cases}
          3yi=x \\
          -xi=3y
        \end{cases} 
        $
        if $y=1$, then $x=3i$ \\
        $
          \Longrightarrow
          V_2=\begin{pmatrix}
            3i \\
            1 \\
          \end{pmatrix}
        $.
        After normalization $V_2$ becomes:
        $
        V_2=\dfrac{1}{\sqrt{10}}\begin{pmatrix}
          3i \\
          1 \\
        \end{pmatrix}
        $
      }

      \textcolor{hwColor}{
        \rule{16cm}{0.4pt}
      }

      \textcolor{hwColor}{
        $
          U=\dfrac{1}{\sqrt{10}}\begin{pmatrix}
            1 & 3i \\
            3i & 1 \\
          \end{pmatrix}
        $, hence 
        $
          \rightarrow 
          U^\dagger=\dfrac{1}{\sqrt{10}}\begin{pmatrix}
            1 & -3i \\
            -3i & 1 \\
          \end{pmatrix}
        $
      }

      \textcolor{hwColor}{
        \rule{16cm}{0.4pt}
      }

      \textcolor{hwColor}{
        $
          HU=\begin{pmatrix}
            10 & 3i \\
            -3i & 2 \\
          \end{pmatrix}.
          \dfrac{1}{\sqrt{10}}\begin{pmatrix}
            1 & 3i \\
            3i & 1 \\
          \end{pmatrix}=\dfrac{1}{\sqrt{10}}\begin{pmatrix}
            1 & 33i \\
            3i & 11 \\
          \end{pmatrix}
        $
      }

      \bigbreak

      \textcolor{hwColor}{
        $
          U^\dagger HU=\dfrac{1}{\sqrt{10}}\begin{pmatrix}
            1 & -3i \\
            -3i & 1 \\
          \end{pmatrix}.\dfrac{1}{\sqrt{10}}\begin{pmatrix}
            1 & 33i \\
            3i & 11 \\
          \end{pmatrix}=\dfrac{1}{10}\begin{pmatrix}
            10 & 0 \\
            0 & 110
          \end{pmatrix}=\begin{pmatrix}
            1 & 0 \\
            0 & 11 \\
          \end{pmatrix}
        $
      }

      \bigbreak

      \textcolor{hwColor}{
        $
          \Longrightarrow
          U^\dagger HU=\Lambda
        $
      }
      
  \end{enumerate}

\end{document}
